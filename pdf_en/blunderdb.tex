%% Generated by Sphinx.
\def\sphinxdocclass{report}
\documentclass[letterpaper,10pt,english]{sphinxmanual}
\ifdefined\pdfpxdimen
   \let\sphinxpxdimen\pdfpxdimen\else\newdimen\sphinxpxdimen
\fi \sphinxpxdimen=.75bp\relax
\ifdefined\pdfimageresolution
    \pdfimageresolution= \numexpr \dimexpr1in\relax/\sphinxpxdimen\relax
\fi
%% let collapsible pdf bookmarks panel have high depth per default
\PassOptionsToPackage{bookmarksdepth=5}{hyperref}

\PassOptionsToPackage{booktabs}{sphinx}
\PassOptionsToPackage{colorrows}{sphinx}

\PassOptionsToPackage{warn}{textcomp}
\usepackage[utf8]{inputenc}
\ifdefined\DeclareUnicodeCharacter
% support both utf8 and utf8x syntaxes
  \ifdefined\DeclareUnicodeCharacterAsOptional
    \def\sphinxDUC#1{\DeclareUnicodeCharacter{"#1}}
  \else
    \let\sphinxDUC\DeclareUnicodeCharacter
  \fi
  \sphinxDUC{00A0}{\nobreakspace}
  \sphinxDUC{2500}{\sphinxunichar{2500}}
  \sphinxDUC{2502}{\sphinxunichar{2502}}
  \sphinxDUC{2514}{\sphinxunichar{2514}}
  \sphinxDUC{251C}{\sphinxunichar{251C}}
  \sphinxDUC{2572}{\textbackslash}
\fi
\usepackage{cmap}
\usepackage[T1]{fontenc}
\usepackage{amsmath,amssymb,amstext}
\usepackage{babel}



\usepackage{tgtermes}
\usepackage{tgheros}
\renewcommand{\ttdefault}{txtt}



\usepackage[Bjarne]{fncychap}
\usepackage{sphinx}

\fvset{fontsize=auto}
\usepackage{geometry}


% Include hyperref last.
\usepackage{hyperref}
% Fix anchor placement for figures with captions.
\usepackage{hypcap}% it must be loaded after hyperref.
% Set up styles of URL: it should be placed after hyperref.
\urlstyle{same}


\usepackage{sphinxmessages}
\setcounter{tocdepth}{1}

\graphicspath{ {./_video_thumbnail/}{./} }
\newcommand{\sphinxcontribyoutube}[3]{\begin{quote}\begin{center}\fbox{\url{#1#2#3}}\end{center}\end{quote}}


\title{blunderDB}
\date{Feb 03, 2025}
\release{0.4.0}
\author{Kevin UNGER \textless{}blunderdb@proton.me\textgreater{}}
\newcommand{\sphinxlogo}{\vbox{}}
\renewcommand{\releasename}{Release}
\makeindex
\begin{document}

\ifdefined\shorthandoff
  \ifnum\catcode`\=\string=\active\shorthandoff{=}\fi
  \ifnum\catcode`\"=\active\shorthandoff{"}\fi
\fi

\pagestyle{empty}
\sphinxmaketitle
\pagestyle{plain}
\sphinxtableofcontents
\pagestyle{normal}
\phantomsection\label{\detokenize{index::doc}}


\sphinxAtStartPar
blunderDB is a software for creating backgammon position databases. Its main strength is to provide a single place where a player can aggregate the positions he or she has encountered (online, in tournaments) and be able to review these positions by filtering them with various filters that can be arbitrarily combined. blunderDB can also be used to create catalogs of reference positions.

\sphinxAtStartPar
The present documentation is structured as follows:
\begin{itemize}
\item {} 
\sphinxAtStartPar
the \sphinxstylestrong{download and installation} section explains how to obtain and run blunderDB.

\item {} 
\sphinxAtStartPar
the \sphinxstylestrong{manual} describes the general functioning of blunderDB and how it should be used.

\item {} 
\sphinxAtStartPar
the \sphinxstylestrong{user guide} is a practical introduction for quickly using blunderDB.

\item {} 
\sphinxAtStartPar
the list of \sphinxstylestrong{commands} as well as the list of \sphinxstylestrong{keyboard shortcuts} enables efficient use of blunderDB.

\item {} 
\sphinxAtStartPar
The \sphinxstylestrong{FAQ} provides answers to the most frequently asked questions.

\end{itemize}


\chapter{Version history}
\label{\detokenize{index:historique-des-versions}}

\begin{savenotes}\sphinxattablestart
\sphinxthistablewithglobalstyle
\centering
\begin{tabular}[t]{\X{5}{32}\X{7}{32}\X{20}{32}}
\sphinxtoprule
\sphinxstyletheadfamily 
\sphinxAtStartPar
Version
&\sphinxstyletheadfamily 
\sphinxAtStartPar
Date
&\sphinxstyletheadfamily 
\sphinxAtStartPar
Cause and/or nature of changes
\\
\sphinxmidrule
\sphinxtableatstartofbodyhook
\sphinxAtStartPar
0.1.0
&
\sphinxAtStartPar
December 31, 2024
&
\sphinxAtStartPar
Beta version
\\
\sphinxhline
\sphinxAtStartPar
0.2.0
&
\sphinxAtStartPar
January 6, 2025
&
\sphinxAtStartPar
Various bug fixes.

\sphinxAtStartPar
Addition of match/TP/GV tables.

\sphinxAtStartPar
Added search filters (moves, cube decisions, date).

\sphinxAtStartPar
Added metadata for positions.

\sphinxAtStartPar
Import/export functionality between blunderDB instances.

\sphinxAtStartPar
Added metadata functionality for databases.

\sphinxAtStartPar
Introduction of version numbers (database and application).
\\
\sphinxhline
\sphinxAtStartPar
0.3.0
&
\sphinxAtStartPar
January 27, 2025
&
\sphinxAtStartPar
Various bug fixes.

\sphinxAtStartPar
Automatically saves the window size.

\sphinxAtStartPar
Imports any comments from XG.
\\
\sphinxhline
\sphinxAtStartPar
0.4.0
&
\sphinxAtStartPar
February 3, 2025
&
\sphinxAtStartPar
Various bug fixes.

\sphinxAtStartPar
Adding an icon for blunderDB.

\sphinxAtStartPar
Filter corrections.

\sphinxAtStartPar
Adding macOS support.
\\
\sphinxbottomrule
\end{tabular}
\sphinxtableafterendhook\par
\sphinxattableend\end{savenotes}


\chapter{Table of contents}
\label{\detokenize{index:sommaire}}
\sphinxstepscope


\section{Download and Installation}
\label{\detokenize{telecharge_install:telechargement-et-installation}}\label{\detokenize{telecharge_install::doc}}
\sphinxAtStartPar
blunderDB is a single executable that requires no installation.

\sphinxAtStartPar
The latest version of blunderDB is available under the MIT license:
\begin{itemize}
\item {} 
\sphinxAtStartPar
for Windows: \sphinxurl{https://github.com/kevung/blunderDB/releases/latest/download/blunderDB-windows-0.4.0.exe}

\item {} 
\sphinxAtStartPar
for Linux: \sphinxurl{https://github.com/kevung/blunderDB/releases/latest/download/blunderDB-linux-0.4.0}

\item {} 
\sphinxAtStartPar
for Mac: \sphinxurl{https://github.com/kevung/blunderDB/releases/latest/download/blunderDB-macos-0.4.0}

\end{itemize}

\begin{sphinxadmonition}{note}{Note:}
\sphinxAtStartPar
blunderDB uses Webview2 for rendering the graphical interface. It is likely that Webview2 is already present natively on your operating system. If not, the first execution of blunderDB will offer to download and install it. No user action is required.
\end{sphinxadmonition}

\begin{sphinxadmonition}{note}{Note:}
\sphinxAtStartPar
On Linux, if blunderDB is not executable after downloading, run the command \sphinxcode{\sphinxupquote{chmod +x ./blunderDB\sphinxhyphen{}linux\sphinxhyphen{}x.y.z}} in a terminal, where x, y, z corresponds to the downloaded version.
\end{sphinxadmonition}

\begin{sphinxadmonition}{warning}{Warning:}
\sphinxAtStartPar
On Windows, it is possible that it may hesitate to run blunderDB. See: \hyperref[\detokenize{annexe_windows_securite:annexe-windows-malware}]{Section \ref{\detokenize{annexe_windows_securite:annexe-windows-malware}}} for an explanation of why and how to bypass any potential blocks.
\end{sphinxadmonition}

\begin{sphinxadmonition}{warning}{Warning:}
\sphinxAtStartPar
On Mac, it is possible that it may have reservations about running blunderDB. See: \sphinxtitleref{annexe\_mac\_malware} to understand why and bypass any potential blocks.
\end{sphinxadmonition}

\sphinxstepscope


\section{Manual}
\label{\detokenize{manuel:manuel}}\label{\detokenize{manuel:id1}}\label{\detokenize{manuel::doc}}
\sphinxAtStartPar
blunderDB is software for creating position databases. The positions are stored in a database represented by a \sphinxstyleemphasis{.db} file.

\sphinxAtStartPar
The main interactions possible with blunderDB are:
\begin{itemize}
\item {} 
\sphinxAtStartPar
add a new position,

\item {} 
\sphinxAtStartPar
modify an existing position,

\item {} 
\sphinxAtStartPar
delete an existing position,

\item {} 
\sphinxAtStartPar
search for one or more positions.

\end{itemize}

\sphinxAtStartPar
To do this, the user switches to dedicated modes for:
\begin{itemize}
\item {} 
\sphinxAtStartPar
navigation and viewing positions (NORMAL mode),

\item {} 
\sphinxAtStartPar
editing positions (EDIT mode),

\item {} 
\sphinxAtStartPar
editing a query to filter positions (COMMAND mode or search window).

\end{itemize}

\sphinxAtStartPar
The user can freely tag positions and annotate them with comments.

\sphinxAtStartPar
In the following sections of the manual, the graphical interface and the main modes of blunderDB are described.


\subsection{Description of the interface}
\label{\detokenize{manuel:description-de-l-interface}}
\sphinxAtStartPar
The interface of blunderDB is composed, from top to bottom, of:
\begin{itemize}
\item {} 
\sphinxAtStartPar
{[}top{]} the toolbar, which gathers all the main operations that can be performed on the database,

\item {} 
\sphinxAtStartPar
{[}in the middle{]} the main display area, which allows for displaying or editing backgammon positions,

\item {} 
\sphinxAtStartPar
{[}at the bottom{]} the status bar, which provides various information about the database or the current position.

\end{itemize}

\sphinxAtStartPar
Panels can be displayed to:
\begin{itemize}
\item {} 
\sphinxAtStartPar
display the analysis data associated with the current position from eXtreme Gammon (XG),

\item {} 
\sphinxAtStartPar
display, add, or modify comments

\end{itemize}

\sphinxAtStartPar
Modal windows can be displayed to:
\begin{itemize}
\item {} 
\sphinxAtStartPar
{[}EDIT mode only{]} set search filters,

\item {} 
\sphinxAtStartPar
display the blunderDB help.

\end{itemize}

\sphinxAtStartPar
The main display area provides the user with:
\begin{itemize}
\item {} 
\sphinxAtStartPar
a board to display or edit a backgammon position,

\item {} 
\sphinxAtStartPar
the level and owner of the cube,

\item {} 
\sphinxAtStartPar
the pip count of each player,

\item {} 
\sphinxAtStartPar
the score of each player,

\item {} 
\sphinxAtStartPar
the dice to play. If no values are displayed on the dice, the position of the dice indicates which player has the turn and that the position is a cube decision.

\end{itemize}

\sphinxAtStartPar
The status bar is structured from left to right with the following information:
\begin{itemize}
\item {} 
\sphinxAtStartPar
the current mode (NORMAL, EDIT, COMMAND),

\item {} 
\sphinxAtStartPar
an informational message related to an operation performed by the user,

\item {} 
\sphinxAtStartPar
the index of the current position, followed by the number of positions in the current library.

\end{itemize}

\begin{sphinxadmonition}{note}{Note:}
\sphinxAtStartPar
In the case of positions resulting from a user search, the number of positions indicated in the status bar corresponds to the number of filtered positions.
\end{sphinxadmonition}


\subsection{NORMAL mode}
\label{\detokenize{manuel:le-mode-normal}}\label{\detokenize{manuel:mode-normal}}
\sphinxAtStartPar
NORMAL mode is the default mode of blunderDB. It is used for:
\begin{itemize}
\item {} 
\sphinxAtStartPar
scrolling through the different positions in the current library,

\item {} 
\sphinxAtStartPar
displaying the analysis information associated with a position.

\item {} 
\sphinxAtStartPar
displaying, adding, and modifying comments on a position.

\end{itemize}

\begin{sphinxadmonition}{tip}{Tip:}
\sphinxAtStartPar
Refer to \hyperref[\detokenize{raccourcis:raccourcis-normal}]{Section \ref{\detokenize{raccourcis:raccourcis-normal}}} for NORMAL mode shortcuts.
\end{sphinxadmonition}


\subsection{EDIT mode}
\label{\detokenize{manuel:le-mode-edit}}\label{\detokenize{manuel:mode-edit}}
\sphinxAtStartPar
EDIT mode allows you to edit a position with the option to either add it to the database or define the type of position to search for. EDIT mode is activated by pressing the \sphinxstyleemphasis{TAB} key. The distribution of checkers, the cube, the score, and the turn can be modified using the mouse (see {\hyperref[\detokenize{guide_utilisateur:guide-edit-position}]{\sphinxcrossref{\DUrole{std}{\DUrole{std-ref}{Edit a position}}}}}).

\begin{sphinxadmonition}{tip}{Tip:}
\sphinxAtStartPar
Refer to \hyperref[\detokenize{raccourcis:raccourcis-edit}]{Section \ref{\detokenize{raccourcis:raccourcis-edit}}} for EDIT mode shortcuts.
\end{sphinxadmonition}


\subsection{COMMAND mode}
\label{\detokenize{manuel:le-mode-command}}\label{\detokenize{manuel:mode-command}}
\sphinxAtStartPar
COMMAND mode allows you to perform all the functionalities of blunderDB available in the graphical interface: general operations on the database, position navigation, displaying analysis and/or comments, searching for positions based on filters… After getting familiar with the interface, it is recommended to gradually use this mode for a powerful and smooth use of blunderDB, especially for position search functionalities.

\sphinxAtStartPar
To switch to COMMAND mode from any other mode, press the \sphinxstyleemphasis{SPACE} key. To submit a query and exit COMMAND mode, press the \sphinxstyleemphasis{ENTER} key.

\sphinxAtStartPar
blunderDB executes the queries sent by the user as long as they are valid and immediately modifies the state of the database if necessary. There are no explicit save actions required from the user.

\begin{sphinxadmonition}{tip}{Tip:}
\sphinxAtStartPar
Refer to \hyperref[\detokenize{cmd_mode:cmd-mode}]{Section \ref{\detokenize{cmd_mode:cmd-mode}}} for the list of available commands in COMMAND mode.
\end{sphinxadmonition}

\sphinxstepscope


\section{User Guide}
\label{\detokenize{guide_utilisateur:guide-utilisateur}}\label{\detokenize{guide_utilisateur:id1}}\label{\detokenize{guide_utilisateur::doc}}
\sphinxAtStartPar
This guide is a practical introduction to pick up quickly blunderDB.


\subsection{Create a new database}
\label{\detokenize{guide_utilisateur:creer-une-nouvelle-base-de-donnees}}
\sphinxAtStartPar
To create a new database, click on the “New Database” button in the toolbar. Choose a path to save the database, as well as a name, and click “Save”.

\begin{sphinxadmonition}{note}{Note:}
\sphinxAtStartPar
The file extension for blunderDB databases is \sphinxstyleemphasis{.db}.
\end{sphinxadmonition}

\begin{sphinxadmonition}{tip}{Tip:}
\sphinxAtStartPar
Keyboard shortcuts: \sphinxstyleemphasis{CTRL\sphinxhyphen{}N}. Command: \sphinxcode{\sphinxupquote{n}}
\end{sphinxadmonition}


\subsection{Open an existing database}
\label{\detokenize{guide_utilisateur:ouvrir-une-base-de-donnee-existante}}
\sphinxAtStartPar
To load an existing database, click on the “Open Database” button in the toolbar. Navigate to the path where the database is located, select the \sphinxstyleemphasis{.db} file, and click “Open.”

\begin{sphinxadmonition}{tip}{Tip:}
\sphinxAtStartPar
Keyboard shortcuts: \sphinxstyleemphasis{CTRL\sphinxhyphen{}O}. Command: \sphinxcode{\sphinxupquote{o}}
\end{sphinxadmonition}


\subsection{Edit a position}
\label{\detokenize{guide_utilisateur:editer-une-position}}\label{\detokenize{guide_utilisateur:guide-edit-position}}
\sphinxAtStartPar
To edit a position, switch to EDIT mode by pressing the \sphinxstyleemphasis{TAB} key. Edit the position with the mouse:
\begin{itemize}
\item {} 
\sphinxAtStartPar
Click on the points to add checkers. A left\sphinxhyphen{}click assigns checkers to player 1. A right\sphinxhyphen{}click assigns checkers to player 2. To insert a prime, click on the starting point, hold down the button, and release on the endpoint. Click on the bar to place checkers in the bar.

\item {} 
\sphinxAtStartPar
To clear the position, double\sphinxhyphen{}click on an empty area outside the board or press the \sphinxstyleemphasis{BACKSPACE} key.

\item {} 
\sphinxAtStartPar
To send the cube to Player 1, left\sphinxhyphen{}click on the cube. To send the cube to Player 2, right\sphinxhyphen{}click on the cube.

\item {} 
\sphinxAtStartPar
To indicate the player who is to move, click on the designated dice area.

\item {} 
\sphinxAtStartPar
To edit the dice, left\sphinxhyphen{}click to increase the value of a die, right\sphinxhyphen{}click to decrease the value of a die. If the dice faces are empty, it means the position is a cube decision.

\item {} 
\sphinxAtStartPar
To edit the players’ score, left\sphinxhyphen{}click to increase the score, right\sphinxhyphen{}click to decrease the score.

\end{itemize}

\begin{sphinxadmonition}{tip}{Tip:}
\sphinxAtStartPar
The input of the position with the mouse for the checkers is done in the same way as in XG.
\end{sphinxadmonition}


\subsection{Add a position to the database}
\label{\detokenize{guide_utilisateur:ajouter-une-position-a-la-base-de-donnees}}
\sphinxAtStartPar
After editing the previous position, blunderDB is in EDIT mode.

\sphinxAtStartPar
To save the previously obtained position, press \sphinxstyleemphasis{CTRL\sphinxhyphen{}S} or click the “Save Position” button in the toolbar.

\begin{sphinxadmonition}{tip}{Tip:}
\sphinxAtStartPar
From EDIT mode, switch to COMMAND mode and execute: \sphinxcode{\sphinxupquote{w}}
\end{sphinxadmonition}


\subsection{Tag a position}
\label{\detokenize{guide_utilisateur:etiqueter-une-position}}
\sphinxAtStartPar
To add a tag \sphinxstyleemphasis{toto} to the current position, switch to COMMAND mode by pressing \sphinxstyleemphasis{SPACE}, type \sphinxcode{\sphinxupquote{\#toto}}, and confirm the command by pressing \sphinxstyleemphasis{ENTER}.


\subsection{Delete a position}
\label{\detokenize{guide_utilisateur:supprimer-une-position}}
\sphinxAtStartPar
To delete the current position from the database, press \sphinxstyleemphasis{Del} or click the “Delete Position” button in the toolbar.

\begin{sphinxadmonition}{tip}{Tip:}
\sphinxAtStartPar
In COMMAND mode, execute \sphinxcode{\sphinxupquote{d}}.
\end{sphinxadmonition}

\begin{sphinxadmonition}{caution}{Caution:}
\sphinxAtStartPar
The deletion of the position is final and does not require any user confirmation.
\end{sphinxadmonition}


\subsection{Import a position from XG}
\label{\detokenize{guide_utilisateur:import-une-position-depuis-xg}}
\sphinxAtStartPar
To import a position directly from XG,
\begin{enumerate}
\sphinxsetlistlabels{\arabic}{enumi}{enumii}{}{.}%
\item {} 
\sphinxAtStartPar
display the position to import in XG and press \sphinxstyleemphasis{CTRL\sphinxhyphen{}C},

\item {} 
\sphinxAtStartPar
open blunderDB and press \sphinxstyleemphasis{CTRL\sphinxhyphen{}V}.

\end{enumerate}


\subsection{Display the analysis of a position imported from XG}
\label{\detokenize{guide_utilisateur:afficher-l-analyse-d-une-position-importee-depuis-xg}}
\sphinxAtStartPar
If a position analyzed by XG has been imported into blunderDB, the XG analysis can be displayed by pressing \sphinxstyleemphasis{CTRL\sphinxhyphen{}L}.

\sphinxAtStartPar
If the position corresponds to a checker decision, the five best moves are displayed on separate lines. For each line, the information provided is in this order: the associated checker move, the normalized equity, the error in equity of the move, the winning chances, gammon, and backgammon for the player, and the winning chances, gammon, and backgammon for the opponent, along with the level of analysis.

\sphinxAtStartPar
If the position corresponds to a cube decision, the cost of each decision is displayed along with the winning chances of the position.


\subsection{Export a position to XG}
\label{\detokenize{guide_utilisateur:exporter-une-position-vers-xg}}
\sphinxAtStartPar
To export a position from blunderDB to XG,
\begin{enumerate}
\sphinxsetlistlabels{\arabic}{enumi}{enumii}{}{.}%
\item {} 
\sphinxAtStartPar
display the position to export in blunderDB and press \sphinxstyleemphasis{CTRL\sphinxhyphen{}C},

\item {} 
\sphinxAtStartPar
open XG and press \sphinxstyleemphasis{CTRL\sphinxhyphen{}V}.

\end{enumerate}


\subsection{View the different positions}
\label{\detokenize{guide_utilisateur:visualiser-les-differentes-positions}}
\sphinxAtStartPar
To view the different positions in the current library, use the \sphinxstyleemphasis{LEFT} and \sphinxstyleemphasis{RIGHT} keys. The \sphinxstyleemphasis{HOME} key allows you to go to the first position, and the \sphinxstyleemphasis{END} key lets you go to the last position.

\sphinxAtStartPar
To display the bearoff on the left, press \sphinxstyleemphasis{CTRL\sphinxhyphen{}LEFT}. To display the bearoff on the right, press \sphinxstyleemphasis{CTRL\sphinxhyphen{}RIGHT}.


\subsection{Search for positions based on criteria}
\label{\detokenize{guide_utilisateur:rechercher-des-positions-selon-des-criteres}}
\sphinxAtStartPar
To search for types of positions,
\begin{itemize}
\item {} 
\sphinxAtStartPar
switch to EDIT mode by pressing \sphinxstyleemphasis{TAB},

\item {} 
\sphinxAtStartPar
edit the structure of the position to search for. blunderDB will filter positions that have at least the entered checker structure. If unsure, to maximize results, clear the position by pressing the \sphinxstyleemphasis{BACKSPACE} key. Edit the cube position and the score if necessary.

\end{itemize}

\sphinxAtStartPar
Method 1 (simple):
\begin{itemize}
\item {} 
\sphinxAtStartPar
Open the search window (\sphinxstyleemphasis{CTRL\sphinxhyphen{}F})

\item {} 
\sphinxAtStartPar
Add and set the search filters

\item {} 
\sphinxAtStartPar
Confirm by clicking on “Search”.

\end{itemize}

\sphinxAtStartPar
Method 2 (advanced):
\begin{itemize}
\item {} 
\sphinxAtStartPar
switch to COMMAND mode by pressing \sphinxstyleemphasis{SPACE},

\item {} 
\sphinxAtStartPar
type \sphinxstyleemphasis{s}, and add any additional filters (for example, \sphinxstyleemphasis{cube} or \sphinxstyleemphasis{score} to consider the cube and score, respectively. See \hyperref[\detokenize{cmd_mode:cmd-filter}]{Section \ref{\detokenize{cmd_mode:cmd-filter}}} for a comprehensive list of available filters).

\item {} 
\sphinxAtStartPar
confirm the request by pressing \sphinxstyleemphasis{ENTER}.

\end{itemize}

\sphinxAtStartPar
The displayed positions are those from the database that meet the search criteria entered by the user.

\sphinxstepscope


\section{List of commands}
\label{\detokenize{cmd_mode:liste-des-commandes}}\label{\detokenize{cmd_mode:cmd-mode}}\label{\detokenize{cmd_mode::doc}}

\subsection{Global operations}
\label{\detokenize{cmd_mode:operations-globales}}\label{\detokenize{cmd_mode:cmd-global}}

\begin{savenotes}\sphinxattablestart
\sphinxthistablewithglobalstyle
\centering
\begin{tabular}[t]{\X{10}{50}\X{40}{50}}
\sphinxtoprule
\sphinxstyletheadfamily 
\sphinxAtStartPar
Command
&\sphinxstyletheadfamily 
\sphinxAtStartPar
Action
\\
\sphinxmidrule
\sphinxtableatstartofbodyhook
\sphinxAtStartPar
new, ne, n
&
\sphinxAtStartPar
Create a new database.
\\
\sphinxhline
\sphinxAtStartPar
open, op, o
&
\sphinxAtStartPar
Open an existing database.
\\
\sphinxhline
\sphinxAtStartPar
quit, q
&
\sphinxAtStartPar
Close blunderDB.
\\
\sphinxhline
\sphinxAtStartPar
help, he, h
&
\sphinxAtStartPar
Open blunderDB help.
\\
\sphinxhline
\sphinxAtStartPar
meta
&
\sphinxAtStartPar
Display database metadata.
\\
\sphinxhline
\sphinxAtStartPar
met
&
\sphinxAtStartPar
Open the Kazaross\sphinxhyphen{}XG2 match equity table.
\\
\sphinxhline
\sphinxAtStartPar
tp2
&
\sphinxAtStartPar
Open the takepoint table with a 2\sphinxhyphen{}cube.
\\
\sphinxhline
\sphinxAtStartPar
tp2\_live
&
\sphinxAtStartPar
Open the takepoint table with a 2\sphinxhyphen{}cube for long race positions.
\\
\sphinxhline
\sphinxAtStartPar
tp2\_last
&
\sphinxAtStartPar
Open the takepoint table with a 2\sphinxhyphen{}cube for last roll positions.
\\
\sphinxhline
\sphinxAtStartPar
tp4
&
\sphinxAtStartPar
Open the takepoint table with a 4\sphinxhyphen{}cube.
\\
\sphinxhline
\sphinxAtStartPar
tp4\_live
&
\sphinxAtStartPar
Open the takepoint table with a 4\sphinxhyphen{}cube for long race positions.
\\
\sphinxhline
\sphinxAtStartPar
tp4\_last
&
\sphinxAtStartPar
Open the takepoint table with a 4\sphinxhyphen{}cube for last roll positions.
\\
\sphinxhline
\sphinxAtStartPar
gv1
&
\sphinxAtStartPar
Open the gammon value table with a 1\sphinxhyphen{}cube.
\\
\sphinxhline
\sphinxAtStartPar
gv2
&
\sphinxAtStartPar
Open the gammon value table with a 2\sphinxhyphen{}cube.
\\
\sphinxhline
\sphinxAtStartPar
gv4
&
\sphinxAtStartPar
Open the gammon value table with a 4\sphinxhyphen{}cube.
\\
\sphinxbottomrule
\end{tabular}
\sphinxtableafterendhook\par
\sphinxattableend\end{savenotes}


\subsection{NORMAL Mode}
\label{\detokenize{cmd_mode:mode-normal}}\label{\detokenize{cmd_mode:cmd-normal}}

\begin{savenotes}\sphinxattablestart
\sphinxthistablewithglobalstyle
\centering
\begin{tabular}[t]{\X{10}{30}\X{20}{30}}
\sphinxtoprule
\sphinxstyletheadfamily 
\sphinxAtStartPar
Command
&\sphinxstyletheadfamily 
\sphinxAtStartPar
Action
\\
\sphinxmidrule
\sphinxtableatstartofbodyhook
\sphinxAtStartPar
import, i
&
\sphinxAtStartPar
Import a position via a text file (txt).
\\
\sphinxhline
\sphinxAtStartPar
delete, del, d
&
\sphinxAtStartPar
Delete the current position.
\\
\sphinxhline
\sphinxAtStartPar
{[}number{]}
&
\sphinxAtStartPar
Go to the specified index position.
\\
\sphinxhline
\sphinxAtStartPar
list, l
&
\sphinxAtStartPar
Show the analysis of the current position.
\\
\sphinxhline
\sphinxAtStartPar
comment, co
&
\sphinxAtStartPar
Show/write comments.
\\
\sphinxhline
\sphinxAtStartPar
\#tag1 tag2 …
&
\sphinxAtStartPar
Tag the current position.
\\
\sphinxhline
\sphinxAtStartPar
e
&
\sphinxAtStartPar
Load all positions from the database.
\\
\sphinxbottomrule
\end{tabular}
\sphinxtableafterendhook\par
\sphinxattableend\end{savenotes}


\subsection{EDIT Mode}
\label{\detokenize{cmd_mode:mode-edit}}\label{\detokenize{cmd_mode:cmd-edit}}

\begin{savenotes}\sphinxattablestart
\sphinxthistablewithglobalstyle
\centering
\begin{tabular}[t]{\X{10}{30}\X{20}{30}}
\sphinxtoprule
\sphinxstyletheadfamily 
\sphinxAtStartPar
Command
&\sphinxstyletheadfamily 
\sphinxAtStartPar
Action
\\
\sphinxmidrule
\sphinxtableatstartofbodyhook
\sphinxAtStartPar
write, wr, w
&
\sphinxAtStartPar
Save the current position.
\\
\sphinxhline
\sphinxAtStartPar
write!, wr!, w!
&
\sphinxAtStartPar
Update the current position.
\\
\sphinxhline
\sphinxAtStartPar
s
&
\sphinxAtStartPar
Search for positions with filters.
\\
\sphinxbottomrule
\end{tabular}
\sphinxtableafterendhook\par
\sphinxattableend\end{savenotes}


\subsection{Search Filters}
\label{\detokenize{cmd_mode:filtres-de-recherche}}\label{\detokenize{cmd_mode:cmd-filter}}
\sphinxAtStartPar
The filters below must be juxtaposed during a search, i.e., after the start of the \sphinxcode{\sphinxupquote{s}} command.

\phantomsection\label{\detokenize{cmd_mode:cmd-filter-pos}}
\begin{sphinxadmonition}{warning}{Warning:}
\sphinxAtStartPar
In the position search, by default, blunderDB takes into account the current checker structure, ignoring the position of the cube, the score, and the dice. To consider the position of the cube, the score, and the dice, it must be explicitly mentioned in the search.
\end{sphinxadmonition}

\begin{sphinxadmonition}{note}{Note:}
\sphinxAtStartPar
blunderDB considers that a backchecker is a checker located between point 24 and point 19.
\end{sphinxadmonition}

\begin{sphinxadmonition}{note}{Note:}
\sphinxAtStartPar
blunderDB considers that the number of checkers in the zone is the number of checkers located between point 12 and point 1.
\end{sphinxadmonition}

\begin{sphinxadmonition}{note}{Note:}
\sphinxAtStartPar
blunderDB considers the outfield to extend between point 18 and point 7.
\end{sphinxadmonition}

\begin{sphinxadmonition}{note}{Note:}
\sphinxAtStartPar
blunderDB considers the jan to extend between point 1 and point 6.
\end{sphinxadmonition}

\begin{sphinxadmonition}{tip}{Tip:}
\sphinxAtStartPar
The parameters for filtering positions can be combined arbitrarily.
\end{sphinxadmonition}


\begin{savenotes}
\sphinxatlongtablestart
\sphinxthistablewithglobalstyle
\makeatletter
  \LTleft \@totalleftmargin plus1fill
  \LTright\dimexpr\columnwidth-\@totalleftmargin-\linewidth\relax plus1fill
\makeatother
\begin{longtable}{\X{10}{30}\X{20}{30}}
\sphinxtoprule
\sphinxstyletheadfamily 
\sphinxAtStartPar
Query
&\sphinxstyletheadfamily 
\sphinxAtStartPar
Action
\\
\sphinxmidrule
\endfirsthead

\multicolumn{2}{c}{\sphinxnorowcolor
    \makebox[0pt]{\sphinxtablecontinued{\tablename\ \thetable{} \textendash{} continued from previous page}}%
}\\
\sphinxtoprule
\sphinxstyletheadfamily 
\sphinxAtStartPar
Query
&\sphinxstyletheadfamily 
\sphinxAtStartPar
Action
\\
\sphinxmidrule
\endhead

\sphinxbottomrule
\multicolumn{2}{r}{\sphinxnorowcolor
    \makebox[0pt][r]{\sphinxtablecontinued{continues on next page}}%
}\\
\endfoot

\endlastfoot
\sphinxtableatstartofbodyhook

\sphinxAtStartPar
cube, cub, cu, c
&
\sphinxAtStartPar
The position checks the cube configuration.
\\
\sphinxhline
\sphinxAtStartPar
score, sco, sc, s
&
\sphinxAtStartPar
The position checks the score.
\\
\sphinxhline
\sphinxAtStartPar
d
&
\sphinxAtStartPar
The position checks the dice or the cube decision.
\\
\sphinxhline
\sphinxAtStartPar
D
&
\sphinxAtStartPar
The position checks the dice roll.
\\
\sphinxhline
\sphinxAtStartPar
p\textgreater{}x
&
\sphinxAtStartPar
The player has at least x pips behind in the race.
\\
\sphinxhline
\sphinxAtStartPar
p\textless{}x
&
\sphinxAtStartPar
The player has at most x pips behind in the race.
\\
\sphinxhline
\sphinxAtStartPar
px,y
&
\sphinxAtStartPar
The player has between x and y pips behind in the race.
\\
\sphinxhline
\sphinxAtStartPar
P\textgreater{}x
&
\sphinxAtStartPar
The player has a race of at least x pips.
\\
\sphinxhline
\sphinxAtStartPar
P\textless{}x
&
\sphinxAtStartPar
The player has a race of at most x pips.
\\
\sphinxhline
\sphinxAtStartPar
Px,y
&
\sphinxAtStartPar
The player has a race between x and y pips.
\\
\sphinxhline
\sphinxAtStartPar
e\textgreater{}x
&
\sphinxAtStartPar
The equity (in millipoints) of the position is greater than x.
\\
\sphinxhline
\sphinxAtStartPar
e\textless{}x
&
\sphinxAtStartPar
The equity (in millipoints) of the position is less than x.
\\
\sphinxhline
\sphinxAtStartPar
ex,y
&
\sphinxAtStartPar
The equity (in millipoints) of the position is between x and y.
\\
\sphinxhline
\sphinxAtStartPar
w\textgreater{}x
&
\sphinxAtStartPar
The player has winning chances greater than x\%.
\\
\sphinxhline
\sphinxAtStartPar
w\textless{}x
&
\sphinxAtStartPar
The player has winning chances less than x\%.
\\
\sphinxhline
\sphinxAtStartPar
wx,y
&
\sphinxAtStartPar
The player has winning chances between x\% and y\%.
\\
\sphinxhline
\sphinxAtStartPar
g\textgreater{}x
&
\sphinxAtStartPar
The player has gammon chances greater than x\%.
\\
\sphinxhline
\sphinxAtStartPar
g\textless{}x
&
\sphinxAtStartPar
The player has gammon chances less than x\%.
\\
\sphinxhline
\sphinxAtStartPar
gx,y
&
\sphinxAtStartPar
The player has gammon chances between x\% and y\%.
\\
\sphinxhline
\sphinxAtStartPar
b\textgreater{}x
&
\sphinxAtStartPar
The player has backgammon chances greater than x\%.
\\
\sphinxhline
\sphinxAtStartPar
b\textless{}x
&
\sphinxAtStartPar
The player has backgammon chances less than x\%.
\\
\sphinxhline
\sphinxAtStartPar
bx,y
&
\sphinxAtStartPar
The player has backgammon chances between x\% and y\%.
\\
\sphinxhline
\sphinxAtStartPar
W\textgreater{}x
&
\sphinxAtStartPar
The opponent has winning chances greater than x\%.
\\
\sphinxhline
\sphinxAtStartPar
W\textless{}x
&
\sphinxAtStartPar
The opponent has winning chances less than x\%.
\\
\sphinxhline
\sphinxAtStartPar
Wx,y
&
\sphinxAtStartPar
The opponent has winning chances between x\% and y\%.
\\
\sphinxhline
\sphinxAtStartPar
G\textgreater{}x
&
\sphinxAtStartPar
The opponent has gammon chances greater than x\%.
\\
\sphinxhline
\sphinxAtStartPar
G\textless{}x
&
\sphinxAtStartPar
The opponent has gammon chances less than x\%.
\\
\sphinxhline
\sphinxAtStartPar
Gx,y
&
\sphinxAtStartPar
The opponent has gammon chances between x\% and y\%.
\\
\sphinxhline
\sphinxAtStartPar
B\textgreater{}x
&
\sphinxAtStartPar
The opponent has backgammon chances greater than x\%.
\\
\sphinxhline
\sphinxAtStartPar
B\textless{}x
&
\sphinxAtStartPar
The opponent has backgammon chances less than x\%.
\\
\sphinxhline
\sphinxAtStartPar
Bx,y
&
\sphinxAtStartPar
The opponent has backgammon chances between x\% and y\%.
\\
\sphinxhline
\sphinxAtStartPar
o\textgreater{}x
&
\sphinxAtStartPar
The player has at least x checkers off.
\\
\sphinxhline
\sphinxAtStartPar
o\textless{}x
&
\sphinxAtStartPar
The player has at most x checkers off.
\\
\sphinxhline
\sphinxAtStartPar
ox,y
&
\sphinxAtStartPar
The player has between x and y checkers off.
\\
\sphinxhline
\sphinxAtStartPar
O\textgreater{}x
&
\sphinxAtStartPar
The opponent has at least x checkers off.
\\
\sphinxhline
\sphinxAtStartPar
O\textless{}x
&
\sphinxAtStartPar
The opponent has at most x checkers off.
\\
\sphinxhline
\sphinxAtStartPar
Ox,y
&
\sphinxAtStartPar
The opponent has between x and y checkers off.
\\
\sphinxhline
\sphinxAtStartPar
k\textgreater{}x
&
\sphinxAtStartPar
The player has at least x backcheckers.
\\
\sphinxhline
\sphinxAtStartPar
k\textless{}x
&
\sphinxAtStartPar
The player has at most x backcheckers.
\\
\sphinxhline
\sphinxAtStartPar
kx,y
&
\sphinxAtStartPar
The player has between x and y backcheckers.
\\
\sphinxhline
\sphinxAtStartPar
K\textgreater{}x
&
\sphinxAtStartPar
The opponent has at least x backcheckers.
\\
\sphinxhline
\sphinxAtStartPar
K\textless{}x
&
\sphinxAtStartPar
The opponent has at most x backcheckers.
\\
\sphinxhline
\sphinxAtStartPar
Kx,y
&
\sphinxAtStartPar
The opponent has between x and y backcheckers.
\\
\sphinxhline
\sphinxAtStartPar
z\textgreater{}x
&
\sphinxAtStartPar
The player has at least x checkers in the zone.
\\
\sphinxhline
\sphinxAtStartPar
z\textless{}x
&
\sphinxAtStartPar
The player has at most x checkers in the zone.
\\
\sphinxhline
\sphinxAtStartPar
zx,y
&
\sphinxAtStartPar
The player has between x and y checkers in the zone.
\\
\sphinxhline
\sphinxAtStartPar
Z\textgreater{}x
&
\sphinxAtStartPar
The opponent has at least x checkers in the zone.
\\
\sphinxhline
\sphinxAtStartPar
Z\textless{}x
&
\sphinxAtStartPar
The opponent has at most x checkers in the zone.
\\
\sphinxhline
\sphinxAtStartPar
Zx,y
&
\sphinxAtStartPar
The opponent has between x and y checkers in the zone.
\\
\sphinxhline
\sphinxAtStartPar
bo\textgreater{}x
&
\sphinxAtStartPar
The player has at least x blots in the outfield.
\\
\sphinxhline
\sphinxAtStartPar
bo\textless{}x
&
\sphinxAtStartPar
The player has at most x blots in the outfield.
\\
\sphinxhline
\sphinxAtStartPar
box,y
&
\sphinxAtStartPar
The player has between x and y blots in the outfield.
\\
\sphinxhline
\sphinxAtStartPar
BO\textgreater{}x
&
\sphinxAtStartPar
The opponent has at least x blots in the outfield.
\\
\sphinxhline
\sphinxAtStartPar
BO\textless{}x
&
\sphinxAtStartPar
The opponent has at most x blots in the outfield.
\\
\sphinxhline
\sphinxAtStartPar
BOx,y
&
\sphinxAtStartPar
The opponent has between x and y blots in the outfield.
\\
\sphinxhline
\sphinxAtStartPar
jb\textgreater{}x
&
\sphinxAtStartPar
The player has at least x blots in the jan.
\\
\sphinxhline
\sphinxAtStartPar
jb\textless{}x
&
\sphinxAtStartPar
The player has at most x blots in the jan.
\\
\sphinxhline
\sphinxAtStartPar
jbx,y
&
\sphinxAtStartPar
The player has between x and y blots in the jan.
\\
\sphinxhline
\sphinxAtStartPar
JB\textgreater{}x
&
\sphinxAtStartPar
The opponent has at least x blots in the jan.
\\
\sphinxhline
\sphinxAtStartPar
JB\textless{}x
&
\sphinxAtStartPar
The opponent has at most x blots in the jan.
\\
\sphinxhline
\sphinxAtStartPar
JBx,y
&
\sphinxAtStartPar
The opponent has between x and y blots in the jan.
\\
\sphinxhline
\sphinxAtStartPar
t’word1;word2;…’
&
\sphinxAtStartPar
The position comments contain at least one of the words.
\\
\sphinxhline
\sphinxAtStartPar
m’pattern1,pattern2,…\textquotesingle{}
&
\sphinxAtStartPar
The best checker moves containing at least one of the patterns.
\\
\sphinxhline
\sphinxAtStartPar
m’ND,DT,DP,…\textquotesingle{}
&
\sphinxAtStartPar
The best cube decisions for No Double/Take, Double Take, Double Pass.
\\
\sphinxhline
\sphinxAtStartPar
T\textgreater{}x
&
\sphinxAtStartPar
Date of position addition after x (YYYY/MM/DD).
\\
\sphinxhline
\sphinxAtStartPar
T\textless{}x
&
\sphinxAtStartPar
Date of position addition before x (YYYY/MM/DD).
\\
\sphinxhline
\sphinxAtStartPar
Tx,y
&
\sphinxAtStartPar
Date of position addition between x and y (YYYY/MM/DD).
\\
\sphinxbottomrule
\end{longtable}
\sphinxtableafterendhook
\sphinxatlongtableend
\end{savenotes}

\begin{sphinxadmonition}{note}{Note:}
\sphinxAtStartPar
Filtering positions based on the dice roll (\sphinxtitleref{D}) necessarily implies filtering positions based on the type of decision (\sphinxtitleref{d}).
\end{sphinxadmonition}

\begin{sphinxadmonition}{note}{Note:}
\sphinxAtStartPar
For the relative difference filter in the race (\sphinxtitleref{p\textgreater{}x}, \sphinxtitleref{p\textless{}x}, \sphinxtitleref{px,y}), the player is behind in the race compared to the opponent if \sphinxtitleref{x\textgreater{}0} and ahead if \sphinxtitleref{x\textless{}0}. For example: \sphinxtitleref{p\textless{}\sphinxhyphen{}10}: the player is at least 10 pips ahead in the race. \sphinxtitleref{p50,70}: the player is between 50 and 70 pips behind in the race.
\end{sphinxadmonition}

\sphinxAtStartPar
For example, the command \sphinxcode{\sphinxupquote{s s c p\sphinxhyphen{}20,\sphinxhyphen{}5 w\textgreater{}60 z\textgreater{}10 K2,3}} filters all positions taking into account the checker structure, the score, and the cube of the edited position where the player has between 20 and 5 pips ahead in the race, with at least 60\% winning chances, at least 10 checkers in the zone, and the opponent has between 2 and 3 backcheckers.

\sphinxstepscope


\section{Keyboard shortcuts}
\label{\detokenize{raccourcis:raccourcis-clavier}}\label{\detokenize{raccourcis:raccourcis}}\label{\detokenize{raccourcis::doc}}

\subsection{General}
\label{\detokenize{raccourcis:general}}\label{\detokenize{raccourcis:raccourcis-generaux}}

\begin{savenotes}\sphinxattablestart
\sphinxthistablewithglobalstyle
\centering
\begin{tabular}[t]{\X{7}{27}\X{20}{27}}
\sphinxtoprule
\sphinxstyletheadfamily 
\sphinxAtStartPar
Shortcut
&\sphinxstyletheadfamily 
\sphinxAtStartPar
Action
\\
\sphinxmidrule
\sphinxtableatstartofbodyhook
\sphinxAtStartPar
CTRL\sphinxhyphen{}N
&
\sphinxAtStartPar
Create a new database.
\\
\sphinxhline
\sphinxAtStartPar
CTRL\sphinxhyphen{}O
&
\sphinxAtStartPar
Open an existing database.
\\
\sphinxhline
\sphinxAtStartPar
CTRL\sphinxhyphen{}Q
&
\sphinxAtStartPar
Close blunderDB.
\\
\sphinxhline
\sphinxAtStartPar
CTRL\sphinxhyphen{}H, ?
&
\sphinxAtStartPar
Show/hide the help.
\\
\sphinxbottomrule
\end{tabular}
\sphinxtableafterendhook\par
\sphinxattableend\end{savenotes}


\subsection{NORMAL mode.}
\label{\detokenize{raccourcis:mode-normal}}\label{\detokenize{raccourcis:raccourcis-normal}}

\begin{savenotes}\sphinxattablestart
\sphinxthistablewithglobalstyle
\centering
\begin{tabular}[t]{\X{7}{27}\X{20}{27}}
\sphinxtoprule
\sphinxstyletheadfamily 
\sphinxAtStartPar
Shortcut
&\sphinxstyletheadfamily 
\sphinxAtStartPar
Action
\\
\sphinxmidrule
\sphinxtableatstartofbodyhook
\sphinxAtStartPar
CTRL\sphinxhyphen{}I
&
\sphinxAtStartPar
Import a position from a text file (txt).
\\
\sphinxhline
\sphinxAtStartPar
CTRL\sphinxhyphen{}C
&
\sphinxAtStartPar
Export a position to the clipboard for import into XG.
\\
\sphinxhline
\sphinxAtStartPar
CTRL\sphinxhyphen{}V
&
\sphinxAtStartPar
Import an XG position.
\\
\sphinxhline
\sphinxAtStartPar
Del
&
\sphinxAtStartPar
Delete the current position.
\\
\sphinxhline
\sphinxAtStartPar
TAB
&
\sphinxAtStartPar
Switch to EDIT mode.
\\
\sphinxhline
\sphinxAtStartPar
SPACE
&
\sphinxAtStartPar
Switch to COMMAND mode.
\\
\sphinxhline
\sphinxAtStartPar
PageUp, h
&
\sphinxAtStartPar
First position.
\\
\sphinxhline
\sphinxAtStartPar
LEFT, k
&
\sphinxAtStartPar
Previous position.
\\
\sphinxhline
\sphinxAtStartPar
RIGHT, j
&
\sphinxAtStartPar
Next position.
\\
\sphinxhline
\sphinxAtStartPar
PageDown, l
&
\sphinxAtStartPar
Last position.
\\
\sphinxhline
\sphinxAtStartPar
CTRL\sphinxhyphen{}LEFT
&
\sphinxAtStartPar
Board orientation to the left.
\\
\sphinxhline
\sphinxAtStartPar
CTRL\sphinxhyphen{}RIGHT
&
\sphinxAtStartPar
Board orientation to the right.
\\
\sphinxhline
\sphinxAtStartPar
CTRL\sphinxhyphen{}K
&
\sphinxAtStartPar
Show the position navigation window.
\\
\sphinxhline
\sphinxAtStartPar
CTRL\sphinxhyphen{}L
&
\sphinxAtStartPar
Show/hide the analysis.
\\
\sphinxhline
\sphinxAtStartPar
CTRL\sphinxhyphen{}P
&
\sphinxAtStartPar
Show/hide the comments.
\\
\sphinxhline
\sphinxAtStartPar
CTRL\sphinxhyphen{}G
&
\sphinxAtStartPar
Show the position metadata.
\\
\sphinxhline
\sphinxAtStartPar
CTRL\sphinxhyphen{}R
&
\sphinxAtStartPar
Reload all the positions from the database.
\\
\sphinxbottomrule
\end{tabular}
\sphinxtableafterendhook\par
\sphinxattableend\end{savenotes}


\subsection{EDIT mode}
\label{\detokenize{raccourcis:mode-edit}}\label{\detokenize{raccourcis:raccourcis-edit}}

\begin{savenotes}\sphinxattablestart
\sphinxthistablewithglobalstyle
\centering
\begin{tabular}[t]{\X{7}{27}\X{20}{27}}
\sphinxtoprule
\sphinxstyletheadfamily 
\sphinxAtStartPar
Shortcut
&\sphinxstyletheadfamily 
\sphinxAtStartPar
Action
\\
\sphinxmidrule
\sphinxtableatstartofbodyhook
\sphinxAtStartPar
TAB
&
\sphinxAtStartPar
Switch to NORMAL mode.
\\
\sphinxhline
\sphinxAtStartPar
SPACE
&
\sphinxAtStartPar
Switch to COMMAND mode.
\\
\sphinxhline
\sphinxAtStartPar
BACKSPACE
&
\sphinxAtStartPar
Clear the current position.
\\
\sphinxhline
\sphinxAtStartPar
CTRL\sphinxhyphen{}S
&
\sphinxAtStartPar
Add a position.
\\
\sphinxhline
\sphinxAtStartPar
CTRL\sphinxhyphen{}U
&
\sphinxAtStartPar
Update a position.
\\
\sphinxhline
\sphinxAtStartPar
CTRL\sphinxhyphen{}F
&
\sphinxAtStartPar
Search for a position.
\\
\sphinxbottomrule
\end{tabular}
\sphinxtableafterendhook\par
\sphinxattableend\end{savenotes}


\subsection{COMMAND mode.}
\label{\detokenize{raccourcis:mode-command}}\label{\detokenize{raccourcis:raccourcis-command}}

\begin{savenotes}\sphinxattablestart
\sphinxthistablewithglobalstyle
\centering
\begin{tabular}[t]{\X{7}{27}\X{20}{27}}
\sphinxtoprule
\sphinxstyletheadfamily 
\sphinxAtStartPar
Shortcut
&\sphinxstyletheadfamily 
\sphinxAtStartPar
Action
\\
\sphinxmidrule
\sphinxtableatstartofbodyhook
\sphinxAtStartPar
ENTER
&
\sphinxAtStartPar
Execute a query.
\\
\sphinxhline
\sphinxAtStartPar
ESC
&
\sphinxAtStartPar
Exit COMMAND mode.
\\
\sphinxhline
\sphinxAtStartPar
BACKSPACE
&
\sphinxAtStartPar
Clear the command. If empty, close COMMAND mode.
\\
\sphinxbottomrule
\end{tabular}
\sphinxtableafterendhook\par
\sphinxattableend\end{savenotes}

\sphinxstepscope


\section{Frequently Asked Questions (FAQ)}
\label{\detokenize{faq:foire-aux-questions}}\label{\detokenize{faq:faq}}\label{\detokenize{faq::doc}}

\subsection{What is the purpose of blunderDB?}
\label{\detokenize{faq:quel-est-l-utilite-de-blunderdb}}
\sphinxAtStartPar
blunderDB allows users to create a personalized database of positions. Its strength lies in not presupposing any classification \sphinxstyleemphasis{a priori}. This gives users the freedom to query positions with great flexibility by combining various criteria (race, structure, cube, score, backcheckers, checkers in the zone, chances of winning/gammon/backgammon, etc.).

\sphinxAtStartPar
Another convenient use of blunderDB is the creation of reference position catalogs. With the ability to tag positions, users can organize all their reference positions in a structured way using a single file. I hope that blunderDB facilitates the sharing of positions between players.


\subsection{What motivated the creation of blunderDB?}
\label{\detokenize{faq:qu-est-ce-qui-a-motive-la-creation-de-blunderdb}}
\sphinxAtStartPar
I used to store interesting positions or blunders in different folders. However, I encountered difficulties in retrieving positions based on criteria that weren’t initially considered in my choice of thematic categories. For example, if the positions were sorted by type of game (race, holding game, blitz, backgame, etc.), how could I retrieve all positions at a certain score or at a given cube level? Additionally, some old positions tended to be forgotten. I wanted a tool that aggregates all my positions without presupposing thematic categories \sphinxstyleemphasis{a priori}, allowing me to ask questions of the database. This type of software is quite common in chess, like ChessBase.


\subsection{How to save the state of the current database?}
\label{\detokenize{faq:comment-sauvegarder-la-base-de-donnees-courante}}
\sphinxAtStartPar
The database is updated immediately upon validation of the request. No explicit saving operation is necessary.


\subsection{Can I modify, copy, share blunderDB?}
\label{\detokenize{faq:puis-je-modifier-copier-partager-blunderdb}}
\sphinxAtStartPar
Yes, absolutely. blunderDB is licensed under the MIT license.


\subsection{What data format does blunderDB use?}
\label{\detokenize{faq:quel-format-de-donnees-utilise-blunderdb}}
\sphinxAtStartPar
The database is a simple SQLite file. In the absence of blunderDB, it can thus be opened with any SQLite file editor.


\subsection{What were the design principles of blunderDB?}
\label{\detokenize{faq:quelles-ont-ete-les-principes-de-conception-de-blunderdb}}
\sphinxAtStartPar
The modal operation of blunderDB (NORMAL, EDIT, COMMAND) is inspired by the very powerful text editor \sphinxhref{https://en.wikipedia.org/wiki/Vim\_(text\_editor)}{Vim}. I wanted blunderDB to be lightweight, standalone, installation\sphinxhyphen{}free, and available on multiple platforms, which led to my choice of the Go programming language and the Svelte library. For database serialization, the file format needed to be cross\sphinxhyphen{}platform and suitable for containing a database. The SQLite file format seemed like the perfect choice.


\subsection{What is the software architecture of blunderDB?}
\label{\detokenize{faq:quel-est-l-architecture-logicielle-de-blunderdb}}\begin{itemize}
\item {} 
\sphinxAtStartPar
The backend is coded in \sphinxhref{https://go.dev/}{Go}. It is responsible for all operations on the SQLite database that stores the positions.

\item {} 
\sphinxAtStartPar
The frontend is coded in \sphinxhref{https://svelte.dev/}{Svelte}. It is responsible for rendering the graphical interface and the Backgammon board.

\item {} 
\sphinxAtStartPar
The application is encapsulated with \sphinxhref{https://wails.io/}{Wails}, enabling the production of native desktop applications that can run on both Windows and Linux.

\item {} 
\sphinxAtStartPar
The database is managed by \sphinxhref{https://www.sqlite.org/}{SQLite}.

\end{itemize}

\sphinxAtStartPar
For more information, see the \sphinxhref{https://github.com/kevung/blunderDB}{blunderDB GitHub repository}.


\subsection{On which platforms does blunderDB run?}
\label{\detokenize{faq:sur-quelles-plateformes-blunderdb-fonctionne-t-il}}
\sphinxAtStartPar
blunderDB runs on Windows and Linux.

\sphinxstepscope


\section{Windows Annex: False Detection of blunderDB as Malware}
\label{\detokenize{annexe_windows_securite:annexe-windows-detection-abusive-de-blunderdb-comme-logiciel-malveillant}}\label{\detokenize{annexe_windows_securite:annexe-windows-malware}}\label{\detokenize{annexe_windows_securite::doc}}
\begin{sphinxadmonition}{note}{Note:}
\sphinxAtStartPar
The following applies to Windows 10 and 11 operating systems.
\end{sphinxadmonition}

\sphinxAtStartPar
Windows now requires software publishing companies or independent software developers to digitally certify their applications, or even distribute them via the Windows Store. It is therefore recommended to turn to external companies to obtain a digital certificate, which costs several hundred euros (see, for example, \sphinxurl{https://learn.microsoft.com/en-us/archive/blogs/ie\_fr/certificats-de-signature-de-code-ev-extended-validation-et-microsoft-smartscreen}).

\sphinxAtStartPar
Since I am offering blunderDB for free, I do not wish to pursue these costly options. Consequently, it is very likely that Windows will warn you of a potential threat or even block the execution of blunderDB entirely. The following sections explain the steps to bypass Windows’ warnings.


\subsection{Windows SmartScreen Warning}
\label{\detokenize{annexe_windows_securite:avertissement-windows-smartscreen}}
\sphinxAtStartPar
After downloading blunderDB, when you run it, Windows may display a warning such as:

\begin{figure}[htbp]
\centering

\noindent\sphinxincludegraphics{{smartscreen_en}.png}
\end{figure}

\sphinxAtStartPar
If you want to allow a specific executable blocked by SmartScreen:
\begin{enumerate}
\sphinxsetlistlabels{\arabic}{enumi}{enumii}{}{.}%
\item {} 
\sphinxAtStartPar
\sphinxstylestrong{Try running the executable}:
\begin{itemize}
\item {} 
\sphinxAtStartPar
When you attempt to launch the executable, SmartScreen may block it and display a warning.

\end{itemize}

\item {} 
\sphinxAtStartPar
\sphinxstylestrong{Click on “More Info”}:
\begin{itemize}
\item {} 
\sphinxAtStartPar
In the SmartScreen warning window, click on \sphinxstylestrong{More Info}.

\end{itemize}

\item {} 
\sphinxAtStartPar
\sphinxstylestrong{Select “Run anyway”}:
\begin{itemize}
\item {} 
\sphinxAtStartPar
If you trust the executable, click \sphinxstylestrong{Run anyway} to bypass the SmartScreen warning for this instance.

\end{itemize}

\end{enumerate}


\subsection{Windows Defender Blocking}
\label{\detokenize{annexe_windows_securite:blocage-windows-defender}}
\sphinxAtStartPar
For certain security settings in Windows, even after bypassing SmartScreen (see the previous section), Windows Defender may prevent the execution of blunderDB with messages such as:

\begin{figure}[htbp]
\centering

\noindent\sphinxincludegraphics{{blunderdb_potential_virus}.png}
\end{figure}

\sphinxAtStartPar
or even:

\begin{figure}[htbp]
\centering

\noindent\sphinxincludegraphics{{threat_found_action_needed}.png}
\end{figure}

\sphinxAtStartPar
or even place it in quarantine.

\sphinxAtStartPar
Windows Defender is known to trigger false positives. This issue is explicitly mentioned in the FAQ on the official Golang website ( \sphinxurl{https://go.dev/doc/faq\#virus} ) or in GitHub tickets for some projects programmed in Go ( \sphinxurl{https://github.com/golang/vscode-go/issues/3182} ).

\sphinxAtStartPar
If you want to prevent Windows Security from scanning blunderDB:
\begin{enumerate}
\sphinxsetlistlabels{\arabic}{enumi}{enumii}{}{.}%
\item {} 
\sphinxAtStartPar
\sphinxstylestrong{Open Windows Security}:
\begin{itemize}
\item {} 
\sphinxAtStartPar
Go to \sphinxstylestrong{Start} and type \sphinxstylestrong{Windows Security}.

\end{itemize}

\end{enumerate}

\begin{figure}[htbp]
\centering

\noindent\sphinxincludegraphics{{win1}.png}
\end{figure}
\begin{enumerate}
\sphinxsetlistlabels{\arabic}{enumi}{enumii}{}{.}%
\setcounter{enumi}{1}
\item {} 
\sphinxAtStartPar
\sphinxstylestrong{Go to “Virus \& Threat Protection”}:
\begin{itemize}
\item {} 
\sphinxAtStartPar
Click on \sphinxstylestrong{Virus \& Threat Protection}.

\end{itemize}

\end{enumerate}

\begin{figure}[htbp]
\centering

\noindent\sphinxincludegraphics{{win2}.png}
\end{figure}
\begin{enumerate}
\sphinxsetlistlabels{\arabic}{enumi}{enumii}{}{.}%
\setcounter{enumi}{2}
\item {} 
\sphinxAtStartPar
\sphinxstylestrong{Manage Settings}:
\begin{itemize}
\item {} 
\sphinxAtStartPar
Scroll down and click on \sphinxstylestrong{Manage settings} under Virus \& Threat Protection settings.

\end{itemize}

\end{enumerate}

\begin{figure}[htbp]
\centering

\noindent\sphinxincludegraphics{{win3}.png}
\end{figure}
\begin{enumerate}
\sphinxsetlistlabels{\arabic}{enumi}{enumii}{}{.}%
\setcounter{enumi}{3}
\item {} 
\sphinxAtStartPar
\sphinxstylestrong{Add or remove exclusions}:
\begin{itemize}
\item {} 
\sphinxAtStartPar
Scroll down to the \sphinxstylestrong{Exclusions} section and click on \sphinxstylestrong{Add or remove exclusions}.

\end{itemize}

\end{enumerate}

\begin{figure}[htbp]
\centering

\noindent\sphinxincludegraphics{{win4}.png}
\end{figure}
\begin{enumerate}
\sphinxsetlistlabels{\arabic}{enumi}{enumii}{}{.}%
\setcounter{enumi}{4}
\item {} 
\sphinxAtStartPar
\sphinxstylestrong{Add an exclusion}:
\begin{itemize}
\item {} 
\sphinxAtStartPar
Click on \sphinxstylestrong{Add an exclusion} and select \sphinxstylestrong{File}. Then, navigate to the executable you want to exclude and select it.

\end{itemize}

\end{enumerate}

\begin{figure}[htbp]
\centering

\noindent\sphinxincludegraphics{{win5}.png}
\end{figure}

\begin{figure}[htbp]
\centering

\noindent\sphinxincludegraphics{{win6}.png}
\end{figure}

\begin{figure}[htbp]
\centering

\noindent\sphinxincludegraphics{{win7}.png}
\end{figure}

\sphinxstepscope


\section{Mac Annex: Possible Blocking of blunderDB}
\label{\detokenize{annexe_mac_securite:annexe-mac-blocage-eventuel-de-blunderdb}}\label{\detokenize{annexe_mac_securite:annexe-mac-malware}}\label{\detokenize{annexe_mac_securite::doc}}
\begin{sphinxadmonition}{note}{Note:}
\sphinxAtStartPar
The following concerns the macOS operating system.
\end{sphinxadmonition}

\sphinxAtStartPar
Mac requires software publishing companies or independent software developers to digitally certify their applications. The developer must enroll in the Apple Developer Program by paying an annual membership fee (\sphinxurl{https://developer.apple.com/support/compare-memberships/}).

\sphinxAtStartPar
Since I am sharing blunderDB for free, I do not wish to pursue these costly options. As a result, Mac will likely warn you of a potential risk or even block the execution of blunderDB entirely. The following sections explain the steps to bypass Mac’s restrictions.


\subsection{Installation of blunderDB}
\label{\detokenize{annexe_mac_securite:installation-de-blunderdb}}
\sphinxAtStartPar
After downloading blunderDB, drag the downloaded file into the Applications section of your Finder. If you have already tried to run blunderDB and Mac warns you of a potential risk, follow the steps below.


\subsection{Authorization to Run blunderDB}
\label{\detokenize{annexe_mac_securite:autorisation-de-l-execution-de-blunderdb}}\begin{enumerate}
\sphinxsetlistlabels{\arabic}{enumi}{enumii}{}{.}%
\item {} 
\sphinxAtStartPar
Open Finder and go to the Applications section.

\item {} 
\sphinxAtStartPar
Find blunderDB and right\sphinxhyphen{}click on it.

\item {} 
\sphinxAtStartPar
Select Open.

\item {} 
\sphinxAtStartPar
A warning window will appear. Click Open.

\item {} 
\sphinxAtStartPar
blunderDB will open, and you can start using it.

\end{enumerate}

\begin{sphinxadmonition}{note}{Note:}
\sphinxAtStartPar
You only need to perform this operation once. Afterward, you can open blunderDB without having to go through these steps.
\end{sphinxadmonition}
\sphinxcontribyoutube{https://youtu.be/}{Ln7XKVFqfUk}{}
\sphinxcontribyoutube{https://youtu.be/}{HkY4iXjxMeI}{}




\chapter{Contact}
\label{\detokenize{index:contacts}}\label{\detokenize{index:id1}}
\sphinxAtStartPar
Author: Kévin Unger \textless{}\sphinxhref{mailto:blunderdb@proton.me}{blunderdb@proton.me}\textgreater{}. You can also find me on Heroes under the username postmanpat.

\sphinxAtStartPar
I initially developed blunderDB for my personal use to help detect patterns in my mistakes. However, it’s very rewarding to receive feedback, especially after spending a lot of time on design, coding, and debugging. So feel free to reach out to share your experiences. All (constructive) feedback is welcome.

\sphinxAtStartPar
Here are several ways to discuss:
\begin{itemize}
\item {} 
\sphinxAtStartPar
Join the Discord server of blunderDB: \sphinxurl{https://discord.gg/DA5PpzM9En}

\item {} 
\sphinxAtStartPar
Email me at \sphinxhref{mailto:blunderdb@proton.me}{blunderdb@proton.me}.

\item {} 
\sphinxAtStartPar
Discuss with me if we meet in a tournament.

\item {} 
\sphinxAtStartPar
On GitHub.
\begin{itemize}
\item {} 
\sphinxAtStartPar
Open an issue: \sphinxurl{https://github.com/kevung/blunderDB/issues}

\item {} 
\sphinxAtStartPar
For bug fixes or improvement suggestions, create a pull request.

\end{itemize}

\end{itemize}


\chapter{Donate}
\label{\detokenize{index:faire-un-don}}
\sphinxAtStartPar
If you appreciate blunderDB and want to support its past and future developments, you can
\begin{itemize}
\item {} 
\sphinxAtStartPar
buy me a drink if we have the pleasure of meeting!

\item {} 
\sphinxAtStartPar
make a small donation via PayPal to the address \sphinxhref{mailto:blunderdb@proton.me}{blunderdb@proton.me}

\end{itemize}


\chapter{Acknowledgments}
\label{\detokenize{index:remerciements}}
\sphinxAtStartPar
I dedicate this little software to my wife Anne\sphinxhyphen{}Claire and our dear daughter Perrine. I would especially like to thank a few friends:
\begin{itemize}
\item {} 
\sphinxAtStartPar
\sphinxstyleemphasis{Tristan Remille}, for introducing me to backgammon with joy and kindness; for showing the way in understanding this wonderful game; and for continuing to encourage me with gentleness and patience in the face of my poor attempts to improve my play.

\item {} 
\sphinxAtStartPar
\sphinxstyleemphasis{Nicolas Harmand}, a cheerful companion for over a decade in great adventures, and a fantastic sparring partner since he has caught the backgammon bug.

\end{itemize}



\renewcommand{\indexname}{Index}
\printindex
\end{document}