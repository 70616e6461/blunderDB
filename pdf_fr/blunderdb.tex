%% Generated by Sphinx.
\def\sphinxdocclass{report}
\documentclass[letterpaper,10pt,french]{sphinxmanual}
\ifdefined\pdfpxdimen
   \let\sphinxpxdimen\pdfpxdimen\else\newdimen\sphinxpxdimen
\fi \sphinxpxdimen=.75bp\relax
\ifdefined\pdfimageresolution
    \pdfimageresolution= \numexpr \dimexpr1in\relax/\sphinxpxdimen\relax
\fi
%% let collapsible pdf bookmarks panel have high depth per default
\PassOptionsToPackage{bookmarksdepth=5}{hyperref}

\PassOptionsToPackage{booktabs}{sphinx}
\PassOptionsToPackage{colorrows}{sphinx}

\PassOptionsToPackage{warn}{textcomp}
\usepackage[utf8]{inputenc}
\ifdefined\DeclareUnicodeCharacter
% support both utf8 and utf8x syntaxes
  \ifdefined\DeclareUnicodeCharacterAsOptional
    \def\sphinxDUC#1{\DeclareUnicodeCharacter{"#1}}
  \else
    \let\sphinxDUC\DeclareUnicodeCharacter
  \fi
  \sphinxDUC{00A0}{\nobreakspace}
  \sphinxDUC{2500}{\sphinxunichar{2500}}
  \sphinxDUC{2502}{\sphinxunichar{2502}}
  \sphinxDUC{2514}{\sphinxunichar{2514}}
  \sphinxDUC{251C}{\sphinxunichar{251C}}
  \sphinxDUC{2572}{\textbackslash}
\fi
\usepackage{cmap}
\usepackage[T1]{fontenc}
\usepackage{amsmath,amssymb,amstext}
\usepackage{babel}



\usepackage{tgtermes}
\usepackage{tgheros}
\renewcommand{\ttdefault}{txtt}



\usepackage[Sonny]{fncychap}
\ChNameVar{\Large\normalfont\sffamily}
\ChTitleVar{\Large\normalfont\sffamily}
\usepackage{sphinx}

\fvset{fontsize=auto}
\usepackage{geometry}


% Include hyperref last.
\usepackage{hyperref}
% Fix anchor placement for figures with captions.
\usepackage{hypcap}% it must be loaded after hyperref.
% Set up styles of URL: it should be placed after hyperref.
\urlstyle{same}


\usepackage{sphinxmessages}
\setcounter{tocdepth}{1}

\graphicspath{ {./_video_thumbnail/}{./} }
\newcommand{\sphinxcontribyoutube}[3]{\begin{quote}\begin{center}\fbox{\url{#1#2#3}}\end{center}\end{quote}}


\title{blunderDB}
\date{05 février 2025}
\release{0.5.3}
\author{Kevin UNGER \textless{}blunderdb@proton.me\textgreater{}}
\newcommand{\sphinxlogo}{\vbox{}}
\renewcommand{\releasename}{Version}
\makeindex
\begin{document}

\ifdefined\shorthandoff
  \ifnum\catcode`\=\string=\active\shorthandoff{=}\fi
  \ifnum\catcode`\"=\active\shorthandoff{"}\fi
\fi

\pagestyle{empty}
\sphinxmaketitle
\pagestyle{plain}
\sphinxtableofcontents
\pagestyle{normal}
\phantomsection\label{\detokenize{index::doc}}


\sphinxAtStartPar
blunderDB est un logiciel pour constituer des bases de données de positions de
backgammon. Sa force principale est de constituer un lieu unique où aggréger
les positions qu’un joueur a pu rencontrer (en ligne, en tournoi) et de pouvoir
réétudier ces positions en les filtrant selon différents filtres combinables
arbitrairement. blunderDB peut également être utilisé pour constituer des
catalogues de positions de référence.

\sphinxAtStartPar
La présente documentation est structurée de la manière suivante:
\begin{itemize}
\item {} 
\sphinxAtStartPar
la section \sphinxstylestrong{téléchargement et installation} explique comment se procurer et
lancer blunderDB.

\item {} 
\sphinxAtStartPar
le \sphinxstylestrong{manuel} décrit le fonctionnement général de blunderDB.

\item {} 
\sphinxAtStartPar
le \sphinxstylestrong{guide utilisateur} est une introduction pratique pour utiliser
rapidement blunderDB.

\item {} 
\sphinxAtStartPar
la liste des \sphinxstylestrong{commandes} ainsi que la liste des \sphinxstylestrong{raccourcis}
clavier permettent une utilisation efficace de blunderDB.

\item {} 
\sphinxAtStartPar
la \sphinxstylestrong{FAQ} fournit quelques réponses aux interrogations les plus fréquentes.

\end{itemize}


\chapter{Historique des versions}
\label{\detokenize{index:historique-des-versions}}

\begin{savenotes}\sphinxattablestart
\sphinxthistablewithglobalstyle
\centering
\begin{tabular}[t]{\X{5}{32}\X{7}{32}\X{20}{32}}
\sphinxtoprule
\sphinxstyletheadfamily 
\sphinxAtStartPar
Version
&\sphinxstyletheadfamily 
\sphinxAtStartPar
Date
&\sphinxstyletheadfamily 
\sphinxAtStartPar
Cause et/ou nature des évolutions
\\
\sphinxmidrule
\sphinxtableatstartofbodyhook
\sphinxAtStartPar
0.1.0
&
\sphinxAtStartPar
31/12/2024
&
\sphinxAtStartPar
Création version beta.
\\
\sphinxhline
\sphinxAtStartPar
0.2.0
&
\sphinxAtStartPar
06/01/2025
&
\sphinxAtStartPar
Résolutions de bugs divers.

\sphinxAtStartPar
Ajout de tables de matchs/TP/GV.

\sphinxAtStartPar
Ajout de filtres de recherche (coups, décision de videau, date).

\sphinxAtStartPar
Ajout de métadonnées sur les positions.

\sphinxAtStartPar
Fonction d’import/export entre instances de blunderDB.

\sphinxAtStartPar
Ajout de fonction de métadonnées sur les bases de données.

\sphinxAtStartPar
Introduction des numéros de versions (base de données et application).
\\
\sphinxhline
\sphinxAtStartPar
0.3.0
&
\sphinxAtStartPar
27/01/2025
&
\sphinxAtStartPar
Résolutions de bugs divers.

\sphinxAtStartPar
Sauvegarde automatiquement le dimensionnement de la fenêtre.

\sphinxAtStartPar
Importe les éventuels commentaires depuis XG.
\\
\sphinxhline
\sphinxAtStartPar
0.4.0
&
\sphinxAtStartPar
03/02/2025
&
\sphinxAtStartPar
Résolutions de bugs divers.

\sphinxAtStartPar
Ajout d’une icone pour blunderDB.

\sphinxAtStartPar
Corrections de filtres.

\sphinxAtStartPar
Ajout du support de MacOS.
\\
\sphinxhline
\sphinxAtStartPar
0.5.0
&
\sphinxAtStartPar
15/02/2025
&
\sphinxAtStartPar
Ajout de nouveaux filtres (miroir, non contact, jan blot, outfield blot).
\\
\sphinxbottomrule
\end{tabular}
\sphinxtableafterendhook\par
\sphinxattableend\end{savenotes}


\chapter{Sommaire}
\label{\detokenize{index:sommaire}}
\sphinxstepscope


\section{Téléchargement et installation}
\label{\detokenize{telecharge_install:telechargement-et-installation}}\label{\detokenize{telecharge_install::doc}}
\sphinxAtStartPar
blunderDB se présente comme un unique exécutable ne nécessitant aucune installation.

\sphinxAtStartPar
La dernière version de blunderDB est disponible en licence MIT:
\begin{itemize}
\item {} 
\sphinxAtStartPar
pour Windows: \sphinxurl{https://github.com/kevung/blunderDB/releases/latest/download/blunderDB-windows-0.5.3.exe}

\item {} 
\sphinxAtStartPar
pour Linux: \sphinxurl{https://github.com/kevung/blunderDB/releases/latest/download/blunderDB-linux-0.5.3}

\item {} 
\sphinxAtStartPar
pour Mac: \sphinxurl{https://github.com/kevung/blunderDB/releases/latest/download/blunderDB-macos-0.5.3}

\end{itemize}

\begin{sphinxadmonition}{note}{Note:}
\sphinxAtStartPar
blunderDB utilise Webview2 pour le rendu de l’interface graphique. Il
y a de fortes chances que Webview2 soit déjà présent nativement sur votre
système d’exploitation. Si ce n’est pas le cas, la première exécution de
blunderDB proposera de le télécharger et de l’installer. Aucune manipulation
de la part de l’utilisateur n’est attendue.
\end{sphinxadmonition}

\begin{sphinxadmonition}{note}{Note:}
\sphinxAtStartPar
Sous Linux, si blunderDB n’est pas exécutable après le
téléchargement, exécuter dans un terminal la commande \sphinxcode{\sphinxupquote{chmod +x
./blunderDB\sphinxhyphen{}linux\sphinxhyphen{}x.y.z}} où x, y, z correspond à la version téléchargée.
\end{sphinxadmonition}

\begin{sphinxadmonition}{warning}{Avertissement:}
\sphinxAtStartPar
Sous Windows, il est possible que ce dernier émette des réticences
à exécuter blunderDB. Voir \hyperref[\detokenize{annexe_windows_securite:annexe-windows-malware}]{Section \ref{\detokenize{annexe_windows_securite:annexe-windows-malware}}} pour comprendre
pourquoi et contourner les éventuels blocages.
\end{sphinxadmonition}

\begin{sphinxadmonition}{warning}{Avertissement:}
\sphinxAtStartPar
Sous Mac, il est possible que ce dernier émette des réticences
à exécuter blunderDB. Voir \hyperref[\detokenize{annexe_mac_securite:annexe-mac-malware}]{Section \ref{\detokenize{annexe_mac_securite:annexe-mac-malware}}} pour comprendre
pourquoi et contourner les éventuels blocages.
\end{sphinxadmonition}

\sphinxstepscope


\section{Manuel}
\label{\detokenize{manuel:manuel}}\label{\detokenize{manuel:id1}}\label{\detokenize{manuel::doc}}
\sphinxAtStartPar
blunderDB est un logiciel pour constituer des bases de données de
positions. Les positions sont stockées dans une base de données représentée par un fichier
\sphinxstyleemphasis{.db}.

\sphinxAtStartPar
Les principales interactions possibles avec blunderDB sont:
\begin{itemize}
\item {} 
\sphinxAtStartPar
ajouter une nouvelle position,

\item {} 
\sphinxAtStartPar
modifier une position existante,

\item {} 
\sphinxAtStartPar
supprimer une position existante,

\item {} 
\sphinxAtStartPar
rechercher une ou plusieurs positions.

\end{itemize}

\sphinxAtStartPar
Pour ce faire, l’utilisateur bascule dans des modes dédiés pour:
\begin{itemize}
\item {} 
\sphinxAtStartPar
la navigation et la visualisation de positions (mode NORMAL),

\item {} 
\sphinxAtStartPar
l’édition de positions (mode EDIT),

\item {} 
\sphinxAtStartPar
l’édition d’une requête pour filtrer des positions (mode COMMAND ou fenêtre de recherche).

\end{itemize}

\sphinxAtStartPar
L’utilisateur peut étiqueter librement les positions à l’aide de tags et les
annoter via des commentaires.

\sphinxAtStartPar
Dans la suite du manuel, il est décrit l’interface graphique ainsi que
les principaux modes de blunderDB.


\subsection{Description de l’interface}
\label{\detokenize{manuel:description-de-l-interface}}
\sphinxAtStartPar
L’interface de blunderDB est constituée de haut en bas par:
\begin{itemize}
\item {} 
\sphinxAtStartPar
{[}en haut{]} la barre d’outils, qui rassemble l’ensemble des principales
opérations réalisables sur la base de données,

\item {} 
\sphinxAtStartPar
{[}au milieu{]} la zone d’affichage principale, qui permet d’afficher ou d’éditer des
positions de backgammon,

\item {} 
\sphinxAtStartPar
{[}en bas{]} la barre d’état, qui présente différentes informations sur la
base de données ou la position courante.

\end{itemize}

\sphinxAtStartPar
Des panneaux peuvent être affichés pour:
\begin{itemize}
\item {} 
\sphinxAtStartPar
afficher les données d’analyse associées à la position courante issues d’eXtreme Gammon (XG),

\item {} 
\sphinxAtStartPar
afficher, ajouter ou modifier des commentaires

\end{itemize}

\sphinxAtStartPar
Des fenêtres modales peuvent s’afficher pour:
\begin{itemize}
\item {} 
\sphinxAtStartPar
{[}mode EDIT uniquement{]} paramétrer les filtres de recherche,

\item {} 
\sphinxAtStartPar
afficher l’aide de blunderDB.

\end{itemize}

\sphinxAtStartPar
La zone d’affichage principale met à disposition à l’utilisateur:
\begin{itemize}
\item {} 
\sphinxAtStartPar
un board afin d’afficher ou d’éditer une position de backgammon,

\item {} 
\sphinxAtStartPar
le niveau et le propriétaire du cube,

\item {} 
\sphinxAtStartPar
le compte de course de chaque joueur,

\item {} 
\sphinxAtStartPar
le score de chaque joueur,

\item {} 
\sphinxAtStartPar
les dés à jouer. Si aucune valeur n’est affichée sur les dés, la
position des dés indique quel joueur a le trait et que la position est
une décision de cube.

\end{itemize}

\sphinxAtStartPar
La barre d’état est structurée de gauche à droite par les informations
suivantes:
\begin{itemize}
\item {} 
\sphinxAtStartPar
le mode courant (NORMAL, EDIT, COMMAND),

\item {} 
\sphinxAtStartPar
un message d’information lié à une opération réalisée par l’utilisateur,

\item {} 
\sphinxAtStartPar
l’index de la position courante, suivi du nombre de positions dans la
bibliothèque courante.

\end{itemize}

\begin{sphinxadmonition}{note}{Note:}
\sphinxAtStartPar
Dans le cas de positions issues d’une recherche par l’utilisateur, le
nombre de positions indiqué dans la barre d’état correspond au nombre de
positions filtrées.
\end{sphinxadmonition}


\subsection{Le mode NORMAL}
\label{\detokenize{manuel:le-mode-normal}}\label{\detokenize{manuel:mode-normal}}
\sphinxAtStartPar
Le mode NORMAL est le mode par défaut de blunderDB. Il est utilisé pour:
\begin{itemize}
\item {} 
\sphinxAtStartPar
faire défiler les différentes positions de la bibliothèque courante,

\item {} 
\sphinxAtStartPar
afficher les informations d’analyse associées à une position,

\item {} 
\sphinxAtStartPar
afficher, ajouter et modifier les commentaires d’une position.

\end{itemize}

\begin{sphinxadmonition}{tip}{Astuce:}
\sphinxAtStartPar
Se référer à la \hyperref[\detokenize{raccourcis:raccourcis-normal}]{Section \ref{\detokenize{raccourcis:raccourcis-normal}}} pour les raccourcis
du mode NORMAL.
\end{sphinxadmonition}


\subsection{Le mode EDIT}
\label{\detokenize{manuel:le-mode-edit}}\label{\detokenize{manuel:mode-edit}}
\sphinxAtStartPar
Le mode EDIT permet d’éditer une position en vue de l’ajouter à
la base de données, ou de définir le type de position à rechercher.
Le mode EDIT est activé en appuyant sur la touche \sphinxstyleemphasis{TAB}.
La distribution des pions, du videau, du score, et du trait peuvent être
modifiés à l’aide de la souris (voir {\hyperref[\detokenize{guide_utilisateur:guide-edit-position}]{\sphinxcrossref{\DUrole{std}{\DUrole{std-ref}{Editer une position}}}}}).

\begin{sphinxadmonition}{tip}{Astuce:}
\sphinxAtStartPar
Se référer à la \hyperref[\detokenize{raccourcis:raccourcis-edit}]{Section \ref{\detokenize{raccourcis:raccourcis-edit}}} pour les raccourcis du mode
EDIT.
\end{sphinxadmonition}


\subsection{Le mode COMMAND}
\label{\detokenize{manuel:le-mode-command}}\label{\detokenize{manuel:mode-command}}
\sphinxAtStartPar
Le mode COMMAND permet de réaliser l’ensemble des fonctionalités de blunderDB
disponibles à l’interface graphique: opérations générales sur la base de
données, navigation de position, affichage de l’analyse et/ou des commentaires,
recherche de positions selon des filtres… Après une première prise en main de
l’interface, il est recommandé de progressivement utiliser ce mode qui permet
une utilisation puissante et fluide de blunderDB, notamment pour les
fonctionnalités de recherche de positions.

\sphinxAtStartPar
Pour basculer dans le mode COMMAND depuis tout autre mode, appuyer sur
la touche \sphinxstyleemphasis{ESPACE}. Pour envoyer une requête et quitter le mode COMMAND,
appuyer sur la touche \sphinxstyleemphasis{ENTREE}.

\sphinxAtStartPar
blunderDB exécute les requêtes envoyées par l’utilisateur sous réserve
qu’elles soient valides et modifie immédiatement l’état de la base de données
le cas échéant. Il n’y a pas d’actions de sauvegarde explicite de la part
de l’utilisateur.

\begin{sphinxadmonition}{tip}{Astuce:}
\sphinxAtStartPar
Se référer à la \hyperref[\detokenize{cmd_mode:cmd-mode}]{Section \ref{\detokenize{cmd_mode:cmd-mode}}} pour la liste de commandes
disponible en mode COMMAND.
\end{sphinxadmonition}

\sphinxstepscope


\section{Guide utilisateur}
\label{\detokenize{guide_utilisateur:guide-utilisateur}}\label{\detokenize{guide_utilisateur:id1}}\label{\detokenize{guide_utilisateur::doc}}
\sphinxAtStartPar
Ce guide est une introduction pratique à blunderDB pour une prise en main
rapide.


\subsection{Créer une nouvelle base de données}
\label{\detokenize{guide_utilisateur:creer-une-nouvelle-base-de-donnees}}
\sphinxAtStartPar
Pour créer une nouvelle base de données, cliquer dans la barre d’outils sur le
bouton « New Database ». Choisir un chemin où enregistrer la base de données,
ainsi qu’un nom et cliquer sur « Save ».

\begin{sphinxadmonition}{note}{Note:}
\sphinxAtStartPar
L’extension des bases de données blunderDB est \sphinxstyleemphasis{.db}.
\end{sphinxadmonition}

\begin{sphinxadmonition}{tip}{Astuce:}
\sphinxAtStartPar
Raccourcis clavier: \sphinxstyleemphasis{CTRL\sphinxhyphen{}N}. Commande: \sphinxcode{\sphinxupquote{n}}
\end{sphinxadmonition}


\subsection{Ouvrir une base de donnée existante}
\label{\detokenize{guide_utilisateur:ouvrir-une-base-de-donnee-existante}}
\sphinxAtStartPar
Pour charger une base de données existante, cliquer dans la barre d’outils sur
le bouton « Open Database ». Choisir le chemin où se trouve la base de données,
choisir le fichier \sphinxstyleemphasis{.db} et cliquer sur « Open ».

\begin{sphinxadmonition}{tip}{Astuce:}
\sphinxAtStartPar
Raccourcis clavier: \sphinxstyleemphasis{CTRL\sphinxhyphen{}O}. Commande: \sphinxcode{\sphinxupquote{o}}
\end{sphinxadmonition}


\subsection{Editer une position}
\label{\detokenize{guide_utilisateur:editer-une-position}}\label{\detokenize{guide_utilisateur:guide-edit-position}}
\sphinxAtStartPar
Pour éditer une position, basculer en mode EDIT à l’aide de la touche \sphinxstyleemphasis{TAB}.
Editer la position à la souris:
\begin{itemize}
\item {} 
\sphinxAtStartPar
cliquer sur les points pour ajouter des pions. Le clic gauche attribue les
pions au joueur 1. Le clic droit attribue les pions au joueur 2. Pour insérer
une prime, cliquer sur le point de départ, maintenir le bouton appuyé,
relacher sur le point d’arrivée. Cliquer sur la barre pour mettre des
pions à la barre.

\item {} 
\sphinxAtStartPar
pour effacer la position, double\sphinxhyphen{}clic sur une zone vide en dehors du board ou
appuyer sur la touche \sphinxstyleemphasis{RETOUR ARRIERE}.

\item {} 
\sphinxAtStartPar
pour envoyer le cube vers le joueur 1, clic gauche sur le cube. Pour envoyer
le cube vers le joueur 2, click droit sur le cube.

\item {} 
\sphinxAtStartPar
pour indiquer le joueur qui a le trait, cliquer à l’emplacement prévu des dés.

\item {} 
\sphinxAtStartPar
pour éditer les dés, clic gauche pour augmenter la valeur d’un dé, clic droit
pour augmenter la valeur d’un dé. Si la face des dés est vide, cela signifie
que la position est une décision de cube.

\item {} 
\sphinxAtStartPar
pour éditer le score des joueurs, clic gauche pour augmenter le score, clic
droit pour réduire le score.

\end{itemize}

\begin{sphinxadmonition}{tip}{Astuce:}
\sphinxAtStartPar
La saisie de la position avec la souris pour les pions se fait de la
même manière que dans XG.
\end{sphinxadmonition}


\subsection{Ajouter une position à la base de données}
\label{\detokenize{guide_utilisateur:ajouter-une-position-a-la-base-de-donnees}}
\sphinxAtStartPar
Après l’édition de la position précédente, blunderDB est dans le mode EDIT.

\sphinxAtStartPar
Pour enregistrer la position obtenue précédemment, faire \sphinxstyleemphasis{CTRL\sphinxhyphen{}S} ou appuiyer
dans la barre d’outils sur le bouton « Save Position ».

\begin{sphinxadmonition}{tip}{Astuce:}
\sphinxAtStartPar
Depuis le mode EDIT, basculer en mode COMMAND et exécuter: \sphinxcode{\sphinxupquote{w}}
\end{sphinxadmonition}


\subsection{Etiqueter une position}
\label{\detokenize{guide_utilisateur:etiqueter-une-position}}
\sphinxAtStartPar
Pour ajouter un tag \sphinxstyleemphasis{toto} à la position courante, basculer en mode COMMAND en appuyant sur \sphinxstyleemphasis{ESPACE},
taper \sphinxcode{\sphinxupquote{\#toto}} et valider la commande en appuyant sur \sphinxstyleemphasis{ENTREE}.


\subsection{Supprimer une position}
\label{\detokenize{guide_utilisateur:supprimer-une-position}}
\sphinxAtStartPar
Pour supprimer la position courante de la base de données, faire \sphinxstyleemphasis{Del} ou
clicker dans la barre d’outils sur le bouton « Delete Position »

\begin{sphinxadmonition}{tip}{Astuce:}
\sphinxAtStartPar
En mode COMMAND, exécuter \sphinxcode{\sphinxupquote{d}}.
\end{sphinxadmonition}

\begin{sphinxadmonition}{caution}{Prudence:}
\sphinxAtStartPar
La suppression de la position est définitive et ne nécessite
aucune confirmation de la part de l’utilisateur.
\end{sphinxadmonition}


\subsection{Import une position depuis XG}
\label{\detokenize{guide_utilisateur:import-une-position-depuis-xg}}
\sphinxAtStartPar
Pour importer une position directement depuis XG,
\begin{enumerate}
\sphinxsetlistlabels{\arabic}{enumi}{enumii}{}{.}%
\item {} 
\sphinxAtStartPar
afficher dans XG la position à importer et appuyer \sphinxstyleemphasis{CTRL\sphinxhyphen{}C},

\item {} 
\sphinxAtStartPar
afficher blunderDB et appuyer \sphinxstyleemphasis{CTRL\sphinxhyphen{}V}.

\end{enumerate}


\subsection{Afficher l’analyse d’une position importée depuis XG}
\label{\detokenize{guide_utilisateur:afficher-l-analyse-d-une-position-importee-depuis-xg}}
\sphinxAtStartPar
Si une position analysée par XG a été importée dans blunderDB, l’analyse de XG
peut être affichée en appuyant \sphinxstyleemphasis{CTRL\sphinxhyphen{}L}.

\sphinxAtStartPar
Si la position correspond à une décision de pions, les cinq meilleurs coups
sont affichés sur des lignes distinctes. Pour chaque ligne, les informations
fournies sont dans cet ordre, le coup de pion associé, l’équité normalisée,
l’erreur en équité du coup, les chances de gain, gammon et backgammon du
joueur, les chances de gain, gammon et backgammon de l’adversaire, le niveau
d’analyse.

\sphinxAtStartPar
Si la position correspond à une décision de cube, le coût de chaque décision
est affiché ainsi que les chances de gain de la position.


\subsection{Exporter une position vers XG}
\label{\detokenize{guide_utilisateur:exporter-une-position-vers-xg}}
\sphinxAtStartPar
Pour exporter une position de blunderDB vers XG,
\begin{enumerate}
\sphinxsetlistlabels{\arabic}{enumi}{enumii}{}{.}%
\item {} 
\sphinxAtStartPar
afficher dans blunderDB la position à exporter et appuyter \sphinxstyleemphasis{CTRL\sphinxhyphen{}C},

\item {} 
\sphinxAtStartPar
afficher XG et appuyer \sphinxstyleemphasis{CTRL\sphinxhyphen{}V}.

\end{enumerate}


\subsection{Visualiser les différentes positions}
\label{\detokenize{guide_utilisateur:visualiser-les-differentes-positions}}
\sphinxAtStartPar
Pour visualiser les différentes positions de la bibliothèque courante, utiliser
les touches \sphinxstyleemphasis{GAUCHE} et \sphinxstyleemphasis{DROITE}. La touche \sphinxstyleemphasis{HOME} permet d’aller à la première
position. La touche \sphinxstyleemphasis{FIN} permet d’aller à la dernière position.

\sphinxAtStartPar
Pour afficher le bearoff à gauche, appuyer \sphinxstyleemphasis{CTRL\sphinxhyphen{}GAUCHE}. Pour afficher le
bearoff à droite, appuyer \sphinxstyleemphasis{CTRL\sphinxhyphen{}DROITE}.


\subsection{Rechercher des positions selon des critères}
\label{\detokenize{guide_utilisateur:rechercher-des-positions-selon-des-criteres}}
\sphinxAtStartPar
Pour rechercher des types de positions,
\begin{itemize}
\item {} 
\sphinxAtStartPar
basculer en mode EDIT en appuyant sur \sphinxstyleemphasis{TAB},

\item {} 
\sphinxAtStartPar
éditer la structure de position à rechercher. blunderDB va filtrer les
positions ayant \sphinxstyleemphasis{a minima} la structure de pions saisie. Dans le
doute, afin d’obtenir le maximum de résultats, effacer la position
en appuyant sur la touche \sphinxstyleemphasis{RETOUR ARRIERE}. Editer si besoin la
position du cube et le score.

\end{itemize}

\sphinxAtStartPar
Méthode 1 (simple):
\begin{itemize}
\item {} 
\sphinxAtStartPar
Ouvrir la fenêtre de recherche (\sphinxstyleemphasis{CTRL\sphinxhyphen{}F})

\item {} 
\sphinxAtStartPar
Ajouter et paramétrer les filtres de recherche

\item {} 
\sphinxAtStartPar
Valider en cliquant sur « Search ».

\end{itemize}

\sphinxAtStartPar
Méthode 2 (avancée):
\begin{itemize}
\item {} 
\sphinxAtStartPar
basculer en mode COMMAND en appuyant sur \sphinxstyleemphasis{ESPACE},

\item {} 
\sphinxAtStartPar
écrire \sphinxstyleemphasis{s}, ajouter d’éventuels filtres supplémentaires (par exemple
\sphinxstyleemphasis{cube} ou \sphinxstyleemphasis{score} pour prendre respectivement en compte le cube et le
score. Voir \hyperref[\detokenize{cmd_mode:cmd-filter}]{Section \ref{\detokenize{cmd_mode:cmd-filter}}} pour une liste exhaustive des
filtres disponibles).

\item {} 
\sphinxAtStartPar
valider la requête en appuyant sur \sphinxstyleemphasis{ENTREE}.

\end{itemize}

\sphinxAtStartPar
Les positions affichées sont celles de la base de données ayant vérifié
les critères de recherche entrés par l’utilisateur.

\sphinxstepscope


\section{Liste des commandes}
\label{\detokenize{cmd_mode:liste-des-commandes}}\label{\detokenize{cmd_mode:cmd-mode}}\label{\detokenize{cmd_mode::doc}}

\subsection{Opérations globales}
\label{\detokenize{cmd_mode:operations-globales}}\label{\detokenize{cmd_mode:cmd-global}}

\begin{savenotes}\sphinxattablestart
\sphinxthistablewithglobalstyle
\centering
\begin{tabular}[t]{\X{10}{50}\X{40}{50}}
\sphinxtoprule
\sphinxstyletheadfamily 
\sphinxAtStartPar
Commande
&\sphinxstyletheadfamily 
\sphinxAtStartPar
Action
\\
\sphinxmidrule
\sphinxtableatstartofbodyhook
\sphinxAtStartPar
new, ne, n
&
\sphinxAtStartPar
Crée une nouvelle base de données.
\\
\sphinxhline
\sphinxAtStartPar
open, op, o
&
\sphinxAtStartPar
Ouvre une base de données existante.
\\
\sphinxhline
\sphinxAtStartPar
quit, q
&
\sphinxAtStartPar
Ferme blunderDB.
\\
\sphinxhline
\sphinxAtStartPar
help, he, h
&
\sphinxAtStartPar
Ouvre l’aide de blunderDB.
\\
\sphinxhline
\sphinxAtStartPar
meta
&
\sphinxAtStartPar
Affiche les métadonnées de la base de données.
\\
\sphinxhline
\sphinxAtStartPar
met
&
\sphinxAtStartPar
Ouvre la table d’équité de match Kazaross\sphinxhyphen{}XG2.
\\
\sphinxhline
\sphinxAtStartPar
tp2
&
\sphinxAtStartPar
Ouvre la table des takepoints avec videau à 2.
\\
\sphinxhline
\sphinxAtStartPar
tp2\_live
&
\sphinxAtStartPar
Ouvre la table des takepoints avec videau à 2 pour les courses longues.
\\
\sphinxhline
\sphinxAtStartPar
tp2\_last
&
\sphinxAtStartPar
Ouvre la table des takepoints avec videau à 2 mort.
\\
\sphinxhline
\sphinxAtStartPar
tp4
&
\sphinxAtStartPar
Ouvre la table des takepoints avec videau à 4.
\\
\sphinxhline
\sphinxAtStartPar
tp4\_live
&
\sphinxAtStartPar
Ouvre la table des takepoints avec videau à 4 pour les courses longues.
\\
\sphinxhline
\sphinxAtStartPar
tp4\_last
&
\sphinxAtStartPar
Ouvre la table des takepoints avec videau à 4 mort.
\\
\sphinxhline
\sphinxAtStartPar
gv1
&
\sphinxAtStartPar
Ouvre la table des valeurs de gammon avec videau à 1.
\\
\sphinxhline
\sphinxAtStartPar
gv2
&
\sphinxAtStartPar
Ouvre la table des valeurs de gammon avec videau à 2.
\\
\sphinxhline
\sphinxAtStartPar
gv4
&
\sphinxAtStartPar
Ouvre la table des valeurs de gammon avec videau à 4.
\\
\sphinxbottomrule
\end{tabular}
\sphinxtableafterendhook\par
\sphinxattableend\end{savenotes}


\subsection{Mode NORMAL}
\label{\detokenize{cmd_mode:mode-normal}}\label{\detokenize{cmd_mode:cmd-normal}}

\begin{savenotes}\sphinxattablestart
\sphinxthistablewithglobalstyle
\centering
\begin{tabular}[t]{\X{10}{30}\X{20}{30}}
\sphinxtoprule
\sphinxstyletheadfamily 
\sphinxAtStartPar
Commande
&\sphinxstyletheadfamily 
\sphinxAtStartPar
Action
\\
\sphinxmidrule
\sphinxtableatstartofbodyhook
\sphinxAtStartPar
import, i
&
\sphinxAtStartPar
Importe une position par fichier texte (txt).
\\
\sphinxhline
\sphinxAtStartPar
delete, del, d
&
\sphinxAtStartPar
Supprime la position courante.
\\
\sphinxhline
\sphinxAtStartPar
{[}number{]}
&
\sphinxAtStartPar
Aller à la position d’indice indiqué.
\\
\sphinxhline
\sphinxAtStartPar
list, l
&
\sphinxAtStartPar
Afficher l’analyse de la position courante.
\\
\sphinxhline
\sphinxAtStartPar
comment, co
&
\sphinxAtStartPar
Afficher/écrire des commentaires.
\\
\sphinxhline
\sphinxAtStartPar
\#tag1 tag2 …
&
\sphinxAtStartPar
Etiqueter la position courante.
\\
\sphinxhline
\sphinxAtStartPar
e
&
\sphinxAtStartPar
Charger toutes les positions de la base de données.
\\
\sphinxbottomrule
\end{tabular}
\sphinxtableafterendhook\par
\sphinxattableend\end{savenotes}


\subsection{Mode EDIT}
\label{\detokenize{cmd_mode:mode-edit}}\label{\detokenize{cmd_mode:cmd-edit}}

\begin{savenotes}\sphinxattablestart
\sphinxthistablewithglobalstyle
\centering
\begin{tabular}[t]{\X{10}{30}\X{20}{30}}
\sphinxtoprule
\sphinxstyletheadfamily 
\sphinxAtStartPar
Commande
&\sphinxstyletheadfamily 
\sphinxAtStartPar
Action
\\
\sphinxmidrule
\sphinxtableatstartofbodyhook
\sphinxAtStartPar
write, wr, w
&
\sphinxAtStartPar
Enregistre la position courante.
\\
\sphinxhline
\sphinxAtStartPar
write!, wr!, w!
&
\sphinxAtStartPar
Mettre à jour la position courante.
\\
\sphinxhline
\sphinxAtStartPar
s
&
\sphinxAtStartPar
Chercher des positions avec des filtres.
\\
\sphinxbottomrule
\end{tabular}
\sphinxtableafterendhook\par
\sphinxattableend\end{savenotes}


\subsection{Filtres de recherche}
\label{\detokenize{cmd_mode:filtres-de-recherche}}\label{\detokenize{cmd_mode:cmd-filter}}
\sphinxAtStartPar
Les filtres ci\sphinxhyphen{}dessous doivent être juxtaposés lors d’une recherche,
c’est\sphinxhyphen{}à\sphinxhyphen{}dire après le début de commande \sphinxcode{\sphinxupquote{s}}.

\phantomsection\label{\detokenize{cmd_mode:cmd-filter-pos}}
\begin{sphinxadmonition}{warning}{Avertissement:}
\sphinxAtStartPar
Dans la recherche de positions, par défaut, blunderDB prend en
compte la structure de pions courante, ignore la position du videau, du
score et des dés. Pour prendre en compte la position du videau, du score,
des dés, il faut le mentionner explicitement dans la recherche.
\end{sphinxadmonition}

\begin{sphinxadmonition}{note}{Note:}
\sphinxAtStartPar
blunderDB considère qu’un pion arriéré (backchecker) est un pion
situé entre le point 24 et le point 19.
\end{sphinxadmonition}

\begin{sphinxadmonition}{note}{Note:}
\sphinxAtStartPar
blunderDB considère que le nombre de pions dans la zone est le nombre
de pions situés entre le point 12 et le point 1.
\end{sphinxadmonition}

\begin{sphinxadmonition}{note}{Note:}
\sphinxAtStartPar
blunderDB considère que l’outfield s’étend entre le point 18 et le point 7.
\end{sphinxadmonition}

\begin{sphinxadmonition}{note}{Note:}
\sphinxAtStartPar
blunderDB considère que le jan s’étend entre le point 1 et le point 6.
\end{sphinxadmonition}

\begin{sphinxadmonition}{tip}{Astuce:}
\sphinxAtStartPar
Les paramètres pour filtrer des positions peuvent être arbitrairement
combinés.
\end{sphinxadmonition}


\begin{savenotes}
\sphinxatlongtablestart
\sphinxthistablewithglobalstyle
\makeatletter
  \LTleft \@totalleftmargin plus1fill
  \LTright\dimexpr\columnwidth-\@totalleftmargin-\linewidth\relax plus1fill
\makeatother
\begin{longtable}{\X{10}{30}\X{20}{30}}
\sphinxtoprule
\sphinxstyletheadfamily 
\sphinxAtStartPar
Requête
&\sphinxstyletheadfamily 
\sphinxAtStartPar
Action
\\
\sphinxmidrule
\endfirsthead

\multicolumn{2}{c}{\sphinxnorowcolor
    \makebox[0pt]{\sphinxtablecontinued{\tablename\ \thetable{} \textendash{} suite de la page précédente}}%
}\\
\sphinxtoprule
\sphinxstyletheadfamily 
\sphinxAtStartPar
Requête
&\sphinxstyletheadfamily 
\sphinxAtStartPar
Action
\\
\sphinxmidrule
\endhead

\sphinxbottomrule
\multicolumn{2}{r}{\sphinxnorowcolor
    \makebox[0pt][r]{\sphinxtablecontinued{suite sur la page suivante}}%
}\\
\endfoot

\endlastfoot
\sphinxtableatstartofbodyhook

\sphinxAtStartPar
cube, cub, cu, c
&
\sphinxAtStartPar
La position vérifie la configuration du cube.
\\
\sphinxhline
\sphinxAtStartPar
score, sco, sc, s
&
\sphinxAtStartPar
La position vérifie le score.
\\
\sphinxhline
\sphinxAtStartPar
d
&
\sphinxAtStartPar
La position vérifie le type de décision (pion ou cube).
\\
\sphinxhline
\sphinxAtStartPar
D
&
\sphinxAtStartPar
La position vérifie le lancer de dés.
\\
\sphinxhline
\sphinxAtStartPar
nc
&
\sphinxAtStartPar
La position est sans contact.
\\
\sphinxhline
\sphinxAtStartPar
M
&
\sphinxAtStartPar
La position ou celle miroir vérifie les filtres.
\\
\sphinxhline
\sphinxAtStartPar
p\textgreater{}x
&
\sphinxAtStartPar
Le joueur a au moins x pips de retard à la course.
\\
\sphinxhline
\sphinxAtStartPar
p\textless{}x
&
\sphinxAtStartPar
Le joueur a au plus x pips de retard à la course.
\\
\sphinxhline
\sphinxAtStartPar
px,y
&
\sphinxAtStartPar
Le joueur a entre x et y pips de retard à la course.
\\
\sphinxhline
\sphinxAtStartPar
P\textgreater{}x
&
\sphinxAtStartPar
Le joueur a une course au moins de x pips.
\\
\sphinxhline
\sphinxAtStartPar
P\textless{}x
&
\sphinxAtStartPar
Le joueur a une course au plus de x pips.
\\
\sphinxhline
\sphinxAtStartPar
Px,y
&
\sphinxAtStartPar
Le joueur a une course entre x et y pips.
\\
\sphinxhline
\sphinxAtStartPar
e\textgreater{}x
&
\sphinxAtStartPar
L’équité (en millipoints) de la position est supérieure à x.
\\
\sphinxhline
\sphinxAtStartPar
e\textless{}x
&
\sphinxAtStartPar
L’équité (en millipoints) de la position est inférieure à x.
\\
\sphinxhline
\sphinxAtStartPar
ex,y
&
\sphinxAtStartPar
L’équité (en millipoints) de la position est comprise entre x et y.
\\
\sphinxhline
\sphinxAtStartPar
w\textgreater{}x
&
\sphinxAtStartPar
Le joueur a des chances de gain supérieures à x \%.
\\
\sphinxhline
\sphinxAtStartPar
w\textless{}x
&
\sphinxAtStartPar
Le joueur a des chances de gain inférieures à x \%.
\\
\sphinxhline
\sphinxAtStartPar
wx,y
&
\sphinxAtStartPar
Le joueur a des chances de gain comprises à x \% et y \%.
\\
\sphinxhline
\sphinxAtStartPar
g\textgreater{}x
&
\sphinxAtStartPar
Le joueur a des chances de gammon supérieures à x \%.
\\
\sphinxhline
\sphinxAtStartPar
g\textless{}x
&
\sphinxAtStartPar
Le joueur a des chances de gammon inférieures à x \%.
\\
\sphinxhline
\sphinxAtStartPar
gx,y
&
\sphinxAtStartPar
Le joueur a des chances de gammon comprises à x \% et y \%.
\\
\sphinxhline
\sphinxAtStartPar
b\textgreater{}x
&
\sphinxAtStartPar
Le joueur a des chances de backgammon supérieures à x \%.
\\
\sphinxhline
\sphinxAtStartPar
b\textless{}x
&
\sphinxAtStartPar
Le joueur a des chances de backgammon inférieures à x \%.
\\
\sphinxhline
\sphinxAtStartPar
bx,y
&
\sphinxAtStartPar
Le joueur a des chances de backgammon comprises à x \% et y \%.
\\
\sphinxhline
\sphinxAtStartPar
W\textgreater{}x
&
\sphinxAtStartPar
L’adversaire a des chances de gain supérieures à x \%.
\\
\sphinxhline
\sphinxAtStartPar
W\textless{}x
&
\sphinxAtStartPar
L’adversaire a des chances de gain inférieures à x \%.
\\
\sphinxhline
\sphinxAtStartPar
Wx,y
&
\sphinxAtStartPar
L’adversaire a des chances de gain comprises à x \% et y \%.
\\
\sphinxhline
\sphinxAtStartPar
G\textgreater{}x
&
\sphinxAtStartPar
L’adversaire a des chances de gammon supérieures à x \%.
\\
\sphinxhline
\sphinxAtStartPar
G\textless{}x
&
\sphinxAtStartPar
L’adversaire a des chances de gammon inférieures à x \%.
\\
\sphinxhline
\sphinxAtStartPar
Gx,y
&
\sphinxAtStartPar
L’adversaire a des chances de gammon comprises à x \% et y \%.
\\
\sphinxhline
\sphinxAtStartPar
B\textgreater{}x
&
\sphinxAtStartPar
L’adversaire a des chances de backgammon supérieures à x \%.
\\
\sphinxhline
\sphinxAtStartPar
B\textless{}x
&
\sphinxAtStartPar
L’adversaire a des chances de backgammon inférieures à x \%.
\\
\sphinxhline
\sphinxAtStartPar
Bx,y
&
\sphinxAtStartPar
L’adversaire a des chances de backgammon comprises à x \% et y \%.
\\
\sphinxhline
\sphinxAtStartPar
o\textgreater{}x
&
\sphinxAtStartPar
Le joueur a au moins x pions sortis.
\\
\sphinxhline
\sphinxAtStartPar
o\textless{}x
&
\sphinxAtStartPar
Le joueur a au plus x pions sortis.
\\
\sphinxhline
\sphinxAtStartPar
ox,y
&
\sphinxAtStartPar
Le joueur a entre x et y pions sortis.
\\
\sphinxhline
\sphinxAtStartPar
O\textgreater{}x
&
\sphinxAtStartPar
L’adversaire a au moins x pions sortis.
\\
\sphinxhline
\sphinxAtStartPar
O\textless{}x
&
\sphinxAtStartPar
L’adversaire a au plus x pions sortis.
\\
\sphinxhline
\sphinxAtStartPar
Ox,y
&
\sphinxAtStartPar
L’adversaire a entre x et y pions sortis.
\\
\sphinxhline
\sphinxAtStartPar
k\textgreater{}x
&
\sphinxAtStartPar
Le joueur a au moins x pions arriérés.
\\
\sphinxhline
\sphinxAtStartPar
k\textless{}x
&
\sphinxAtStartPar
Le joueur a au plus x pions arriérés.
\\
\sphinxhline
\sphinxAtStartPar
kx,y
&
\sphinxAtStartPar
Le joueur a entre x et y pions arriérés.
\\
\sphinxhline
\sphinxAtStartPar
K\textgreater{}x
&
\sphinxAtStartPar
L’adversaire a au moins x pions arriérés.
\\
\sphinxhline
\sphinxAtStartPar
K\textless{}x
&
\sphinxAtStartPar
L’adversaire a au plus x pions arriérés.
\\
\sphinxhline
\sphinxAtStartPar
Kx,y
&
\sphinxAtStartPar
L’adversaire a entre x et y pions arriérés.
\\
\sphinxhline
\sphinxAtStartPar
z\textgreater{}x
&
\sphinxAtStartPar
Le joueur a au moins x pions dans la zone.
\\
\sphinxhline
\sphinxAtStartPar
z\textless{}x
&
\sphinxAtStartPar
Le joueur a au plus x pions dans la zone.
\\
\sphinxhline
\sphinxAtStartPar
zx,y
&
\sphinxAtStartPar
Le joueur a entre x et y pions dans la zone.
\\
\sphinxhline
\sphinxAtStartPar
Z\textgreater{}x
&
\sphinxAtStartPar
L’adversaire a au moins x pions dans la zone.
\\
\sphinxhline
\sphinxAtStartPar
Z\textless{}x
&
\sphinxAtStartPar
L’adversaire a au plus x pions dans la zone.
\\
\sphinxhline
\sphinxAtStartPar
Zx,y
&
\sphinxAtStartPar
L’adversaire a entre x et y pions dans la zone.
\\
\sphinxhline
\sphinxAtStartPar
bo\textgreater{}x
&
\sphinxAtStartPar
Le joueur a au moins x blots dans l’outfield.
\\
\sphinxhline
\sphinxAtStartPar
bo\textless{}x
&
\sphinxAtStartPar
Le joueur a au plus x blots dans l’outfield.
\\
\sphinxhline
\sphinxAtStartPar
box,y
&
\sphinxAtStartPar
Le joueur a entre x et y blots dans l’outfield.
\\
\sphinxhline
\sphinxAtStartPar
BO\textgreater{}x
&
\sphinxAtStartPar
L’adversaire a au moins x blots dans l’outfield.
\\
\sphinxhline
\sphinxAtStartPar
BO\textless{}x
&
\sphinxAtStartPar
L’adversaire a au plus x blots dans l’outfield.
\\
\sphinxhline
\sphinxAtStartPar
BOx,y
&
\sphinxAtStartPar
L’adversaire a entre x et y blots dans l’outfield.
\\
\sphinxhline
\sphinxAtStartPar
jb\textgreater{}x
&
\sphinxAtStartPar
Le joueur a au moins x blots dans le jan.
\\
\sphinxhline
\sphinxAtStartPar
jb\textless{}x
&
\sphinxAtStartPar
Le joueur a au plus x blots dans le jan.
\\
\sphinxhline
\sphinxAtStartPar
jbx,y
&
\sphinxAtStartPar
Le joueur a entre x et y blots dans le jan.
\\
\sphinxhline
\sphinxAtStartPar
JB\textgreater{}x
&
\sphinxAtStartPar
L’adversaire a au moins x blots dans le jan.
\\
\sphinxhline
\sphinxAtStartPar
JB\textless{}x
&
\sphinxAtStartPar
L’adversaire a au plus x blots dans le jan.
\\
\sphinxhline
\sphinxAtStartPar
JBx,y
&
\sphinxAtStartPar
L’adversaire a entre x et y blots dans le jan.
\\
\sphinxhline
\sphinxAtStartPar
t’mot1;mot2;…”
&
\sphinxAtStartPar
Les commentaires de la position contiennent au moins un des mots.
\\
\sphinxhline
\sphinxAtStartPar
m’motif1,motif2,…\textquotesingle{}
&
\sphinxAtStartPar
Les meilleurs coups de pions contenant au moins un des motifs.
\\
\sphinxhline
\sphinxAtStartPar
m’ND,DT,DP,…\textquotesingle{}
&
\sphinxAtStartPar
Les meilleurs décisions de videau de No Double/Take, Double Take, Double Pass.
\\
\sphinxhline
\sphinxAtStartPar
T\textgreater{}x
&
\sphinxAtStartPar
Date d’ajout de la position après x (AAAA/MM/JJ).
\\
\sphinxhline
\sphinxAtStartPar
T\textless{}x
&
\sphinxAtStartPar
Date d’ajout de la position avant x (AAAA/MM/JJ).
\\
\sphinxhline
\sphinxAtStartPar
Tx,y
&
\sphinxAtStartPar
Date d’ajout de la position entre x et y (AAAA/MM/JJ).
\\
\sphinxbottomrule
\end{longtable}
\sphinxtableafterendhook
\sphinxatlongtableend
\end{savenotes}

\begin{sphinxadmonition}{note}{Note:}
\sphinxAtStartPar
Filtrer les positions en fonction du lancer de dés (\sphinxtitleref{D}) implique \sphinxstyleemphasis{a
fortiori} de filtrer les positions en fonction du type de décision (\sphinxtitleref{d}).
\end{sphinxadmonition}

\begin{sphinxadmonition}{note}{Note:}
\sphinxAtStartPar
Pour le filtre de différence relative à la course (\sphinxtitleref{p\textgreater{}x}, \sphinxtitleref{p\textless{}x},
\sphinxtitleref{px,y}), le joueur est en retard à la course par rapport à l’adversaire si
\sphinxtitleref{x\textgreater{}0} et en avance si \sphinxtitleref{x\textless{}0}. Exemple: \sphinxtitleref{p\textless{}\sphinxhyphen{}10} : le joueur a au moins 10 pips
d’avance à la course. \sphinxtitleref{p50,70} : le joueur a entre 50 et 70 pips de retard à
la course.
\end{sphinxadmonition}

\sphinxAtStartPar
Par exemple, la commande \sphinxcode{\sphinxupquote{s s c p\sphinxhyphen{}20,\sphinxhyphen{}5 w\textgreater{}60 z\textgreater{}10 K2,3}} filtre toutes les
positions en prenant en compte la structure des pions, le score et le cube
de la position éditée où le joueur a entre 20 et 5 pips d’avance à la
course, avec au moins 60\% de chances de gain, au moins 10 pions dans la
zone, et l’adversaire a entre 2 et 3 pions arriérés.


\subsection{Commandes diverses}
\label{\detokenize{cmd_mode:commandes-diverses}}\label{\detokenize{cmd_mode:cmd-misc}}

\begin{savenotes}\sphinxattablestart
\sphinxthistablewithglobalstyle
\centering
\begin{tabular}[t]{\X{10}{50}\X{40}{50}}
\sphinxtoprule
\sphinxstyletheadfamily 
\sphinxAtStartPar
Commande
&\sphinxstyletheadfamily 
\sphinxAtStartPar
Action
\\
\sphinxmidrule
\sphinxtableatstartofbodyhook
\sphinxAtStartPar
clear, cl
&
\sphinxAtStartPar
Efface l’historique des commandes.
\\
\sphinxhline
\sphinxAtStartPar
migrate\_from\_1\_0\_to\_1\_1
&
\sphinxAtStartPar
Migre la base de données de la version 1.0 à la version 1.1.
\\
\sphinxbottomrule
\end{tabular}
\sphinxtableafterendhook\par
\sphinxattableend\end{savenotes}

\sphinxstepscope


\section{Raccourcis clavier}
\label{\detokenize{raccourcis:raccourcis-clavier}}\label{\detokenize{raccourcis:raccourcis}}\label{\detokenize{raccourcis::doc}}

\subsection{Général}
\label{\detokenize{raccourcis:general}}\label{\detokenize{raccourcis:raccourcis-generaux}}

\begin{savenotes}\sphinxattablestart
\sphinxthistablewithglobalstyle
\centering
\begin{tabular}[t]{\X{7}{27}\X{20}{27}}
\sphinxtoprule
\sphinxstyletheadfamily 
\sphinxAtStartPar
Raccourci
&\sphinxstyletheadfamily 
\sphinxAtStartPar
Action
\\
\sphinxmidrule
\sphinxtableatstartofbodyhook
\sphinxAtStartPar
CTRL\sphinxhyphen{}N
&
\sphinxAtStartPar
Créer une nouvelle base de données.
\\
\sphinxhline
\sphinxAtStartPar
CTRL\sphinxhyphen{}O
&
\sphinxAtStartPar
Ouvrir une base de données existante.
\\
\sphinxhline
\sphinxAtStartPar
CTRL\sphinxhyphen{}Q
&
\sphinxAtStartPar
Fermer blunderDB.
\\
\sphinxhline
\sphinxAtStartPar
CTRL\sphinxhyphen{}H, ?
&
\sphinxAtStartPar
Afficher/cacher l’aide.
\\
\sphinxbottomrule
\end{tabular}
\sphinxtableafterendhook\par
\sphinxattableend\end{savenotes}


\subsection{Mode NORMAL}
\label{\detokenize{raccourcis:mode-normal}}\label{\detokenize{raccourcis:raccourcis-normal}}

\begin{savenotes}\sphinxattablestart
\sphinxthistablewithglobalstyle
\centering
\begin{tabular}[t]{\X{7}{27}\X{20}{27}}
\sphinxtoprule
\sphinxstyletheadfamily 
\sphinxAtStartPar
Raccourci
&\sphinxstyletheadfamily 
\sphinxAtStartPar
Action
\\
\sphinxmidrule
\sphinxtableatstartofbodyhook
\sphinxAtStartPar
CTRL\sphinxhyphen{}I
&
\sphinxAtStartPar
Importer une position par fichier texte (txt).
\\
\sphinxhline
\sphinxAtStartPar
CTRL\sphinxhyphen{}C
&
\sphinxAtStartPar
Exporter une position dans le presse\sphinxhyphen{}papier en vue d’un import dans XG.
\\
\sphinxhline
\sphinxAtStartPar
CTRL\sphinxhyphen{}V
&
\sphinxAtStartPar
Importer une position XG.
\\
\sphinxhline
\sphinxAtStartPar
Del
&
\sphinxAtStartPar
Supprimer la position courante.
\\
\sphinxhline
\sphinxAtStartPar
TAB
&
\sphinxAtStartPar
Basculer en mode EDIT.
\\
\sphinxhline
\sphinxAtStartPar
ESPACE
&
\sphinxAtStartPar
Basculer en mode COMMAND.
\\
\sphinxhline
\sphinxAtStartPar
PageUp, h
&
\sphinxAtStartPar
Première position.
\\
\sphinxhline
\sphinxAtStartPar
GAUCHE, k
&
\sphinxAtStartPar
Position précédente.
\\
\sphinxhline
\sphinxAtStartPar
DROITE, j
&
\sphinxAtStartPar
Position suivante.
\\
\sphinxhline
\sphinxAtStartPar
PageDown, l
&
\sphinxAtStartPar
Dernière position.
\\
\sphinxhline
\sphinxAtStartPar
CTRL\sphinxhyphen{}GAUCHE
&
\sphinxAtStartPar
Orientation du board à gauche.
\\
\sphinxhline
\sphinxAtStartPar
CTRL\sphinxhyphen{}DROITE
&
\sphinxAtStartPar
Orientation du board à droite.
\\
\sphinxhline
\sphinxAtStartPar
CTRL\sphinxhyphen{}K
&
\sphinxAtStartPar
Afficher la fenêtre de navigation de positions.
\\
\sphinxhline
\sphinxAtStartPar
CTRL\sphinxhyphen{}L
&
\sphinxAtStartPar
Afficher/cacher l’analyse.
\\
\sphinxhline
\sphinxAtStartPar
CTRL\sphinxhyphen{}P
&
\sphinxAtStartPar
Afficher/cacher les commentaires.
\\
\sphinxhline
\sphinxAtStartPar
CTRL\sphinxhyphen{}G
&
\sphinxAtStartPar
Afficher les métadonnées de la position.
\\
\sphinxhline
\sphinxAtStartPar
CTRL\sphinxhyphen{}R
&
\sphinxAtStartPar
Recharger toutes les positions de la base de données.
\\
\sphinxbottomrule
\end{tabular}
\sphinxtableafterendhook\par
\sphinxattableend\end{savenotes}


\subsection{Mode EDIT}
\label{\detokenize{raccourcis:mode-edit}}\label{\detokenize{raccourcis:raccourcis-edit}}

\begin{savenotes}\sphinxattablestart
\sphinxthistablewithglobalstyle
\centering
\begin{tabular}[t]{\X{7}{27}\X{20}{27}}
\sphinxtoprule
\sphinxstyletheadfamily 
\sphinxAtStartPar
Raccourci
&\sphinxstyletheadfamily 
\sphinxAtStartPar
Action
\\
\sphinxmidrule
\sphinxtableatstartofbodyhook
\sphinxAtStartPar
TAB
&
\sphinxAtStartPar
Basculer en mode NORMAL.
\\
\sphinxhline
\sphinxAtStartPar
ESPACE
&
\sphinxAtStartPar
Basculer en mode COMMAND.
\\
\sphinxhline
\sphinxAtStartPar
RETOUR ARRIERE
&
\sphinxAtStartPar
Effacer la position courante.
\\
\sphinxhline
\sphinxAtStartPar
CTRL\sphinxhyphen{}S
&
\sphinxAtStartPar
Ajouter une position.
\\
\sphinxhline
\sphinxAtStartPar
CTRL\sphinxhyphen{}U
&
\sphinxAtStartPar
Mettre à jour une position.
\\
\sphinxhline
\sphinxAtStartPar
CTRL\sphinxhyphen{}F
&
\sphinxAtStartPar
Rechercher une position.
\\
\sphinxbottomrule
\end{tabular}
\sphinxtableafterendhook\par
\sphinxattableend\end{savenotes}


\subsection{Mode COMMAND}
\label{\detokenize{raccourcis:mode-command}}\label{\detokenize{raccourcis:raccourcis-command}}

\begin{savenotes}\sphinxattablestart
\sphinxthistablewithglobalstyle
\centering
\begin{tabular}[t]{\X{7}{27}\X{20}{27}}
\sphinxtoprule
\sphinxstyletheadfamily 
\sphinxAtStartPar
Raccourci
&\sphinxstyletheadfamily 
\sphinxAtStartPar
Action
\\
\sphinxmidrule
\sphinxtableatstartofbodyhook
\sphinxAtStartPar
ENTREE
&
\sphinxAtStartPar
Exécuter une requête.
\\
\sphinxhline
\sphinxAtStartPar
ESC
&
\sphinxAtStartPar
Quitter le mode COMMAND.
\\
\sphinxhline
\sphinxAtStartPar
RETOUR ARRIERE
&
\sphinxAtStartPar
Effacer la commande. Si vide, fermer le mode COMMAND
\\
\sphinxhline
\sphinxAtStartPar
HAUT
&
\sphinxAtStartPar
Parcourir l’historique des commandes vers le haut.
\\
\sphinxhline
\sphinxAtStartPar
BAS
&
\sphinxAtStartPar
Parcourir l’historique des commandes vers le bas.
\\
\sphinxbottomrule
\end{tabular}
\sphinxtableafterendhook\par
\sphinxattableend\end{savenotes}

\sphinxstepscope


\section{Foire aux questions}
\label{\detokenize{faq:foire-aux-questions}}\label{\detokenize{faq:faq}}\label{\detokenize{faq::doc}}

\subsection{Quel est l’utilité de blunderDB?}
\label{\detokenize{faq:quel-est-l-utilite-de-blunderdb}}
\sphinxAtStartPar
blunderDB permet de constituer une base de données personalisée de
positions. Sa force est de ne présupposer aucune classification \sphinxstyleemphasis{a
priori}. L’utilisateur a ainsi la liberté d’interroger les
positions avec une grande flexibilité en combinant à sa guise
différents critères (course, structure, cube, score, pions arriérés,
pions dans la zone, chances de gain/gammon/backgammon, …).

\sphinxAtStartPar
Une autre utilisation commode de blunderDB est la constitution de catalogues de
positions de référence. Avec la possibilité d’étiqueter des positions,
l’utilisateur peut rassembler l’ensemble de ses positions de référence de
manière structurée à l’aide d’un unique fichier. Je souhaite que blunderDB
facilite le partage de positions entre joueurs.


\subsection{Qu’est ce qui a motivé la création de blunderDB?}
\label{\detokenize{faq:qu-est-ce-qui-a-motive-la-creation-de-blunderdb}}
\sphinxAtStartPar
J’avais l’habitude de stocker dans différents dossiers des positions
intéressantes ou des blunders. Toutefois, je rencontrais des difficultés à
retrouver des positions selon des critères n’ayant pas été prévus initialement
par mon choix de catégories de thématiques. Par exemple, si les positions ont
été triées selon le type de jeu (course, holding game, blitz, backgame, …),
comment récupérer toutes les positions à un certain score? ou à un niveau de
cube donné? Enfin, certaines vieilles positions avaient tendance à tomber dans
l’oubli. Je voulais un outil qui aggrège toutes mes positions et qui ne
présuppose pas \sphinxstyleemphasis{a priori} de catégories thématiques, et ensuite pouvoir poser
des questions à la base de données. Avec cette approche souple, de nouveaux
filtres peuvent être ajoutés sans casser l’organisation des positions. Ce type
de logiciel est tout à fait courant aux échecs, comme ChessBase.


\subsection{Comment sauvegarder la base de données courante?}
\label{\detokenize{faq:comment-sauvegarder-la-base-de-donnees-courante}}
\sphinxAtStartPar
La base de données est modifiée immédiatement après exécution des requêtes.
Aucune opération de sauvegarde explicite est nécessaire.


\subsection{Puis\sphinxhyphen{}je modifier, copier, partager blunderDB?}
\label{\detokenize{faq:puis-je-modifier-copier-partager-blunderdb}}
\sphinxAtStartPar
Oui, tout à fait (et c’est même encouragé!). blunderDB est sous licence MIT.


\subsection{Quel format de données utilise blunderDB?}
\label{\detokenize{faq:quel-format-de-donnees-utilise-blunderdb}}
\sphinxAtStartPar
La base de données est un simple fichier Sqlite. En l’absence de
blunderDB, elle peut ainsi s’ouvrir avec tout éditeur de fichier sqlite.


\subsection{Quelles ont été les principes de conception de blunderDB?}
\label{\detokenize{faq:quelles-ont-ete-les-principes-de-conception-de-blunderdb}}
\sphinxAtStartPar
Le fonctionnement modal de blunderDB (NORMAL, EDIT, COMMAND) s’inspire du très
puissant éditeur de texte \sphinxhref{https://en.wikipedia.org/wiki/Vim\_(text\_editor)}{Vim}. Je souhaitais blunderDB
léger, autonome, sans installation et disponible pour différentes plateformes,
d’où mon choix du langage Go et de la bibliothèque Svelte. Pour la
sérialisation de la base de données, le format de fichiers doit être
multi\sphinxhyphen{}plateforme et adapté pour contenir une base de données. Le format de
fichier sqlite semblait tout indiqué.


\subsection{Quel est l’architecture logicielle de blunderDB?}
\label{\detokenize{faq:quel-est-l-architecture-logicielle-de-blunderdb}}\begin{itemize}
\item {} 
\sphinxAtStartPar
Le backend est codé en \sphinxhref{https://go.dev/}{Go}. Il est en charge de
l’ensemble des opérations sur la base de données Sqlite qui stocke les
positions.

\item {} 
\sphinxAtStartPar
Le frontend est codé en \sphinxhref{https://svelte.dev/}{Svelte}. Il est en charge du
rendu de l’interface graphique et du board de Backgammon.

\item {} 
\sphinxAtStartPar
L’application est encapsulée avec \sphinxhref{https://wails.io/}{Wails}, permettant la
production d’applications Desktop natives, déclinables sous Windows et Linux.

\item {} 
\sphinxAtStartPar
La base de données est gérée par \sphinxhref{https://www.sqlite.org/}{Sqlite}.

\end{itemize}

\sphinxAtStartPar
Pour plus d’informations, voir le \sphinxhref{https://github.com/kevung/blunderDB}{dépôt Github de blunderDB}.


\subsection{Sur quelles plateformes blunderDB fonctionne\sphinxhyphen{}t’il?}
\label{\detokenize{faq:sur-quelles-plateformes-blunderdb-fonctionne-t-il}}
\sphinxAtStartPar
blunderDB fonctionne sur Windows, Linux et Mac.


\subsection{D’où vient l’icône de blunderDB?}
\label{\detokenize{faq:d-ou-vient-l-icone-de-blunderdb}}
\sphinxAtStartPar
L’icône de blunderDB est l’émoticône « goggling » de la série \sphinxhref{https://commons.wikimedia.org/wiki/SMirC}{SMirC}.

\sphinxstepscope


\section{Annexe Windows : Détection abusive de blunderDB comme logiciel malveillant}
\label{\detokenize{annexe_windows_securite:annexe-windows-detection-abusive-de-blunderdb-comme-logiciel-malveillant}}\label{\detokenize{annexe_windows_securite:annexe-windows-malware}}\label{\detokenize{annexe_windows_securite::doc}}
\begin{sphinxadmonition}{note}{Note:}
\sphinxAtStartPar
Ce qui suit concerne les systèmes d’exploitation Windows 10 et 11.
\end{sphinxadmonition}

\sphinxAtStartPar
Windows requiert aujourd’hui de la part de sociétés d’édition logicielle ou
d’éditeurs logiciel indépendants de certifier numériquement leurs applications
voire de distribuer via le Windows Store. Il est alors préconisé de faire appel
à des sociétés extérieures pour obtenir un certificat numérique au prix de
plusieurs centaines d’euros (voir par exemple
\sphinxurl{https://learn.microsoft.com/en-us/archive/blogs/ie\_fr/certificats-de-signature-de-code-ev-extended-validation-et-microsoft-smartscreen}
).

\sphinxAtStartPar
Partageant blunderDB gratuitement, je ne souhaite pas m’orienter vers ces
possibilités onéreuses. Par conséquent, il est fort probable que Windows vous
avertisse d’un potentiel danger, voire bloque complètement l’exécution de
blunderDB. Les sections suivantes expliquent les opérations à réaliser pour
passer outre les réticences de Windows.


\subsection{Avertissement Windows SmartScreen}
\label{\detokenize{annexe_windows_securite:avertissement-windows-smartscreen}}
\sphinxAtStartPar
Après téléchargement de blunderDB, lors de son exécution, il est possible que
Windows affiche un avertissement du type

\begin{figure}[htbp]
\centering

\noindent\sphinxincludegraphics{{smartscreen_en}.png}
\end{figure}

\sphinxAtStartPar
Si vous souhaitez autoriser un exécutable spécifique bloqué par SmartScreen :
\begin{enumerate}
\sphinxsetlistlabels{\arabic}{enumi}{enumii}{}{.}%
\item {} 
\sphinxAtStartPar
\sphinxstylestrong{Essayer d’exécuter l’exécutable} :
\begin{itemize}
\item {} 
\sphinxAtStartPar
Lorsque vous essayez de lancer l’exécutable, SmartScreen peut le bloquer
et afficher un avertissement.

\end{itemize}

\item {} 
\sphinxAtStartPar
\sphinxstylestrong{Cliquer sur « Informations supplémentaires »} :
\begin{itemize}
\item {} 
\sphinxAtStartPar
Dans la fenêtre d’avertissement de SmartScreen, cliquez sur \sphinxstylestrong{Informations
supplémentaires}.

\end{itemize}

\item {} 
\sphinxAtStartPar
\sphinxstylestrong{Sélectionner « Exécuter quand même »} :
\begin{itemize}
\item {} 
\sphinxAtStartPar
Si vous faites confiance à l’exécutable, cliquez sur \sphinxstylestrong{Exécuter quand
même} pour contourner l’avertissement SmartScreen pour cette instance.

\end{itemize}

\end{enumerate}


\subsection{Blocage Windows Defender}
\label{\detokenize{annexe_windows_securite:blocage-windows-defender}}
\sphinxAtStartPar
Pour certains paramétrages sécurité de Windows, il arrive que malgré le
déblocage de SmartScreen (voir section plus précédente), Windows Defender
puisse empêcher l’exécution de blunderDB avec des messages du type

\begin{figure}[htbp]
\centering

\noindent\sphinxincludegraphics{{blunderdb_potential_virus}.png}
\end{figure}

\sphinxAtStartPar
ou encore

\begin{figure}[htbp]
\centering

\noindent\sphinxincludegraphics{{threat_found_action_needed}.png}
\end{figure}

\sphinxAtStartPar
voire le placer en quarantaine.

\sphinxAtStartPar
Windows Defender est connu pour déclencher des faux positifs. Ce problème est
explicitement mentionné dans la FAQ du site officiel de Golang (
\sphinxurl{https://go.dev/doc/faq\#virus} ) ou dans des tickets Github de certains projets
programmés en Go ( \sphinxurl{https://github.com/golang/vscode-go/issues/3182} ).

\sphinxAtStartPar
Si vous souhaitez empêcher la Sécurité Windows d’analyser blunderDB :
\begin{enumerate}
\sphinxsetlistlabels{\arabic}{enumi}{enumii}{}{.}%
\item {} 
\sphinxAtStartPar
\sphinxstylestrong{Ouvrir la Sécurité Windows} :
\begin{itemize}
\item {} 
\sphinxAtStartPar
Allez dans \sphinxstylestrong{Démarrer} et tapez \sphinxstylestrong{Sécurité Windows}.

\end{itemize}

\end{enumerate}

\begin{figure}[htbp]
\centering

\noindent\sphinxincludegraphics{{win1}.png}
\end{figure}
\begin{enumerate}
\sphinxsetlistlabels{\arabic}{enumi}{enumii}{}{.}%
\setcounter{enumi}{1}
\item {} 
\sphinxAtStartPar
\sphinxstylestrong{Aller à « Protection contre les virus et menaces »} :
\begin{itemize}
\item {} 
\sphinxAtStartPar
Cliquez sur \sphinxstylestrong{Protection contre les virus et menaces}.

\end{itemize}

\end{enumerate}

\begin{figure}[htbp]
\centering

\noindent\sphinxincludegraphics{{win2}.png}
\end{figure}
\begin{enumerate}
\sphinxsetlistlabels{\arabic}{enumi}{enumii}{}{.}%
\setcounter{enumi}{2}
\item {} 
\sphinxAtStartPar
\sphinxstylestrong{Gérer les paramètres} :
\begin{itemize}
\item {} 
\sphinxAtStartPar
Faites défiler vers le bas et cliquez sur \sphinxstylestrong{Gérer les paramètres} sous Paramètres de protection contre les virus et menaces.

\end{itemize}

\end{enumerate}

\begin{figure}[htbp]
\centering

\noindent\sphinxincludegraphics{{win3}.png}
\end{figure}
\begin{enumerate}
\sphinxsetlistlabels{\arabic}{enumi}{enumii}{}{.}%
\setcounter{enumi}{3}
\item {} 
\sphinxAtStartPar
\sphinxstylestrong{Ajouter ou supprimer des exclusions} :
\begin{itemize}
\item {} 
\sphinxAtStartPar
Faites défiler jusqu’à la section \sphinxstylestrong{Exclusions} et cliquez sur \sphinxstylestrong{Ajouter ou supprimer des exclusions}.

\end{itemize}

\end{enumerate}

\begin{figure}[htbp]
\centering

\noindent\sphinxincludegraphics{{win4}.png}
\end{figure}
\begin{enumerate}
\sphinxsetlistlabels{\arabic}{enumi}{enumii}{}{.}%
\setcounter{enumi}{4}
\item {} 
\sphinxAtStartPar
\sphinxstylestrong{Ajouter une exclusion} :
\begin{itemize}
\item {} 
\sphinxAtStartPar
Cliquez sur \sphinxstylestrong{Ajouter une exclusion} et sélectionnez \sphinxstylestrong{Fichier}. Naviguez ensuite jusqu’à
l’exécutable que vous souhaitez exclure et sélectionnez\sphinxhyphen{}le.

\end{itemize}

\end{enumerate}

\begin{figure}[htbp]
\centering

\noindent\sphinxincludegraphics{{win5}.png}
\end{figure}

\begin{figure}[htbp]
\centering

\noindent\sphinxincludegraphics{{win6}.png}
\end{figure}

\begin{figure}[htbp]
\centering

\noindent\sphinxincludegraphics{{win7}.png}
\end{figure}

\sphinxstepscope


\section{Annexe Mac : Blocage éventuel de blunderDB}
\label{\detokenize{annexe_mac_securite:annexe-mac-blocage-eventuel-de-blunderdb}}\label{\detokenize{annexe_mac_securite:annexe-mac-malware}}\label{\detokenize{annexe_mac_securite::doc}}
\begin{sphinxadmonition}{note}{Note:}
\sphinxAtStartPar
Ce qui suit concerne le système d’exploitation MacOS.
\end{sphinxadmonition}

\sphinxAtStartPar
Mac requiert de la part de sociétés d’édition logicielle ou
d’éditeurs logiciel indépendants de certifier numériquement leurs applications.
Le développeur doit souscrire au Apple Developper Program en payant une adhésion annuelle (\sphinxurl{https://developer.apple.com/support/compare-memberships/}).

\sphinxAtStartPar
Partageant blunderDB gratuitement, je ne souhaite pas m’orienter vers ces
possibilités onéreuses. Par conséquent, il est fort probable que Mac vous
avertisse d’un potentiel danger, voire bloque complètement l’exécution de
blunderDB. Les sections suivantes expliquent les opérations à réaliser pour
passer outre les réticences de Mac.


\subsection{Installation de blunderDB}
\label{\detokenize{annexe_mac_securite:installation-de-blunderdb}}
\sphinxAtStartPar
Après avoir téléchargé blunderDB, glissez le fichier téléchargé dans la section
Applications de votre Finder. Si vous avez déjà essayé d’exécuter blunderDB et
que Mac vous avertit d’un potentiel danger, suivez les étapes suivantes.


\subsection{Autorisation de l’exécution de blunderDB}
\label{\detokenize{annexe_mac_securite:autorisation-de-l-execution-de-blunderdb}}\begin{enumerate}
\sphinxsetlistlabels{\arabic}{enumi}{enumii}{}{.}%
\item {} 
\sphinxAtStartPar
Ouvrez le Finder et allez dans la section Applications.

\item {} 
\sphinxAtStartPar
Trouvez blunderDB et faites un clic droit.

\item {} 
\sphinxAtStartPar
Sélectionnez Ouvrir.

\item {} 
\sphinxAtStartPar
Une fenêtre d’avertissement s’ouvre. Cliquez sur Ouvrir.

\item {} 
\sphinxAtStartPar
blunderDB s’ouvre et vous pouvez l’utiliser.

\end{enumerate}

\begin{sphinxadmonition}{note}{Note:}
\sphinxAtStartPar
Vous n’avez à réaliser cette opération qu’une seule fois. Par la
suite, vous pourrez ouvrir blunderDB sans avoir à passer par ces étapes.
\end{sphinxadmonition}

\sphinxstepscope


\section{Annexe: Schéma de la base de données}
\label{\detokenize{annexe_db_scheme:annexe-schema-de-la-base-de-donnees}}\label{\detokenize{annexe_db_scheme:annexe-db-migration}}\label{\detokenize{annexe_db_scheme::doc}}

\subsection{Version 1.0.0}
\label{\detokenize{annexe_db_scheme:version-1-0-0}}
\sphinxAtStartPar
La version 1.0.0 de la base de données contient les tables suivantes :
\begin{itemize}
\item {} 
\sphinxAtStartPar
\sphinxstylestrong{position} : Stocke les positions avec les colonnes \sphinxtitleref{id} (clé primaire) et \sphinxtitleref{state} (état de la position en format JSON).

\item {} 
\sphinxAtStartPar
\sphinxstylestrong{analysis} : Stocke les analyses des positions avec les colonnes \sphinxtitleref{id} (clé primaire), \sphinxtitleref{position\_id} (clé étrangère vers \sphinxtitleref{position}), et \sphinxtitleref{data} (données de l’analyse en format JSON).

\item {} 
\sphinxAtStartPar
\sphinxstylestrong{comment} : Stocke les commentaires associés aux positions avec les colonnes \sphinxtitleref{id} (clé primaire), \sphinxtitleref{position\_id} (clé étrangère vers \sphinxtitleref{position}), et \sphinxtitleref{text} (texte du commentaire).

\item {} 
\sphinxAtStartPar
\sphinxstylestrong{metadata} : Stocke les métadonnées de la base de données avec les colonnes \sphinxtitleref{key} (clé primaire) et \sphinxtitleref{value} (valeur associée à la clé).

\end{itemize}


\subsection{Version 1.1.0}
\label{\detokenize{annexe_db_scheme:version-1-1-0}}
\sphinxAtStartPar
La version 1.1.0 de la base de données ajoute la table suivante :
\begin{itemize}
\item {} 
\sphinxAtStartPar
\sphinxstylestrong{command\_history} : Stocke l’historique des commandes avec les colonnes \sphinxtitleref{id} (clé primaire), \sphinxtitleref{command} (texte de la commande), et \sphinxtitleref{timestamp} (date et heure de l’exécution de la commande).

\end{itemize}

\sphinxAtStartPar
Les autres tables restent inchangées par rapport à la version 1.0.0.

\sphinxAtStartPar
Pour migrer la base de données de la version 1.0.0 à la version 1.1.0, exécutez la commande \sphinxcode{\sphinxupquote{migrate\_from\_1\_0\_to\_1\_1}} dans blunderDB.
\sphinxcontribyoutube{https://youtu.be/}{Ln7XKVFqfUk}{}
\sphinxcontribyoutube{https://youtu.be/}{HkY4iXjxMeI}{}




\chapter{Contacts}
\label{\detokenize{index:contacts}}\label{\detokenize{index:id1}}
\sphinxAtStartPar
Auteur: Kévin Unger \textless{}\sphinxhref{mailto:blunderdb@proton.me}{blunderdb@proton.me}\textgreater{}.
Vous pouvez aussi me trouver sur Heroes sous le pseudo postmanpat.

\sphinxAtStartPar
J’ai développé blunderDB initialement pour mon usage personnel afin de
pouvoir détecter des motifs dans mes erreurs. Mais il est très agréable
d’avoir un retour surtout quand on a dépensé un paquet d’heures de
conception, codage, débuggage… Aussi n’hésitez pas à m’écrire pour
faire part de votre retour d’expérience. Tous les retours (constructifs)
sont bienvenus.

\sphinxAtStartPar
Voici plusieurs manières de discuter:
\begin{itemize}
\item {} 
\sphinxAtStartPar
rejoindre le serveur Discord de blunderDB: \sphinxurl{https://discord.gg/DA5PpzM9En}

\item {} 
\sphinxAtStartPar
m’écrire un mail à \sphinxhref{mailto:blunderdb@proton.me}{blunderdb@proton.me},

\item {} 
\sphinxAtStartPar
discuter avec moi, si on se retrouve dans un tournoi,

\item {} 
\sphinxAtStartPar
sur Github,
\begin{itemize}
\item {} 
\sphinxAtStartPar
ouvrir un ticket: \sphinxurl{https://github.com/kevung/blunderDB/issues}

\item {} 
\sphinxAtStartPar
pour des corrections de bugs ou des propositions d’amélioration,
créer une pull request.

\end{itemize}

\end{itemize}


\chapter{Faire un don}
\label{\detokenize{index:faire-un-don}}
\sphinxAtStartPar
Si vous appréciez blunderDB et que vous voulez soutenir les développements passés et futurs, vous pouvez
\begin{itemize}
\item {} 
\sphinxAtStartPar
me payer un verre si on a le plaisir de se rencontrer!

\item {} 
\sphinxAtStartPar
faire un petit don par PayPal à l’adresse \sphinxhref{mailto:blunderdb@proton.me}{blunderdb@proton.me}

\end{itemize}


\chapter{Remerciements}
\label{\detokenize{index:remerciements}}
\sphinxAtStartPar
Je dédie ce petit logiciel à ma compagne Anne\sphinxhyphen{}Claire et notre tendre
fille Perrine. Je tiens à remercier tout particulièrement quelques amis:
\begin{itemize}
\item {} 
\sphinxAtStartPar
\sphinxstyleemphasis{Tristan Remille}, de m’avoir initié au backgammon avec joie et
bienveillance; de montrer la Voie dans la compréhension de ce
merveilleux jeu; de continuer à m’encourager malgré mes piètres
tentatives de mieux jouer.

\item {} 
\sphinxAtStartPar
\sphinxstyleemphasis{Nicolas Harmand}, joyeux camarade depuis maintenant plus d’une dizaine
d’années dans de chouettes aventures, et un fantastique partenaire de jeu
depuis qu’il a choppé le virus du backgammon.

\end{itemize}



\renewcommand{\indexname}{Index}
\printindex
\end{document}